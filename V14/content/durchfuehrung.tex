\newpage
\section{Durchführung}
\label{sec:Durchführung}

\begin{figure}[htb]
  \centering
  \includegraphics[width=10.0cm]{pictures/Würfel.pdf}
  \caption{Würfelpositionen und der schematische Strahlenverlauf für die Messungen.}
  \label{fig:wuerfel}
\end{figure}

\begin{figure}[htb]
  \centering
  \includegraphics[height=7.0cm]{content/pictures/Aufbau.png}
  \caption{Darstellung des verwendeten Versuchsaufbaus.\cite{anleitung}}
  \label{fig:Aufbau}
\end{figure}
Zu Beginn des Versuchs wird der $\ce{^137Cs}$-Strahler durch den Assistenten, in den Versuchsaufbau aus Abbildung \ref{fig:Aufbau} eingebaut. 
Dieser befindet sich links im Aufbau. 
Durch ein Loch wird ein Strahl auf den Würfel fokussiert.
Der Würfel befindet sich dabei auf einem drehbaren Halterung, die auch wie in Abbildung \ref{fig:Aufbau} dargestellt in eine Richtung verschieben lässt.
Die Höhe der Halterung ist fest eingestellt, sodass die mittlere Ebene vermessen wird.
Nach durch dringen des Würfels wird die Strahlung im Szintillationsdetektor (\ce{NaJ}-Detektor) nachgewiesen.
Anschließend wird eine Nullmessung, dass bedeutet keine Probe im Strahlengang aufgenommen.
Hierzu wird das Programm \textit{Maestro} verwendet, welches auf dem bereitgestellten Computer installiert ist.
Der Multichannel-Analyser ist an den Photo-Multiplier des Szintillationsdetektor angeschlossen, welcher die $\gamma$-Strahlung nachweist.
Anschließend wird nur ein Aluminiumgehäuse, welches die Würfel im Inneren zusammen hält vermessen. Dabei ist darauf zu achten, dass für einen statistischen Fehler
von weniger als \SI{3}{\percent} eine Zählrate von \num{1112} erreicht werden muss, was sich nach der Poissonverteilung ergibt. Es werden zwölf Messungen mit einem Strahlenverlauf wie in Abbildung \ref{fig:wuerfel} dargestellt, durchgeführt.
Nun werden noch zwei Vollwürfel untersucht, welche ebenfalls durch ein Aluminiumgehäuse geschützt sind. Dazu werden für beide nur vier Messungen benötigt,
da die jeweils anderen äquivalent sind.
Zuletzt wird noch ein Würfel, welcher aus \num{27} Einzelwürfeln besteht genauer untersucht. Der Aufbau lässt aber nur die Vermessung der mittleren Ebene zu, hier werden 
wieder alle zwölf Messungen durchgeführt.
\FloatBarrier
\newpage