\textbf{Diskrimator:} Erlaubt die Betrachtung eines definierten Energieauschnitts, der Rest wird diskriminiert. (Erzeugt Spannung proportional zum eingang von Impulsen pro Zeiteinheit)
\textbf{Multichannelanalyser - Vielkanalanalysator:} Sortiert Impulse nach Ihrer Höhe ein und setzt den Zähler für diese Höhe um eins nach oben.
\textbf{Least-Square-Fit - Methode der kleinesten Quadrate:} Dabei wird zu einer Datenpunktwolke eine Kurve gesucht, die möglichst nahe an den Datenpunkten verläuft.
\textbf{Szintillationsdetektor:} Durch Stoßprozesse entstehen in den Szintillatoren Lichtquanten, welche durch einen Photomultiplier verstärkt und regestriert werden. Energie $\approx$ Lichtmenge, Intensität $\approx$ Anzahl Szintillationen pro Zeiteinheit 
\textbf{Photomultiplier:} Lichtquant löst Elektron aus und wird lawinenartig verstärkt (Zick-Zack durch Beschleunigungskaskade) Amplitude ist abhängig von einfallender Energie
Anzahl Pulse pro Zeiteiheit Maß für die Intensität

 \begin{equation*}
   A = \begin{pmatrix}
              1 & 0 & 0 & 1 & 0 & 0 & 1 & 0 & 0 \\
              0 & 1 & 0 & 0 & 1 & 0 & 0 & 1 & 0 \\
              0 & 0 & 1 & 0 & 0 & 1 & 0 & 0 & 1 \\
              1 & 1 & 1 & 0 & 0 & 0 & 0 & 0 & 0 \\
              0 & 0 & 0 & 1 & 1 & 1 & 0 & 0 & 0 \\
              0 & 0 & 0 & 0 & 0 & 0 & 1 & 1 & 1 \\
              0 & \sqrt{2} & 0 & \sqrt{2} & 0 & 0 & 0 & 0 & 0 \\
              0 & 0 & \sqrt{2} & 0 & \sqrt{2} & 0 & \sqrt{2} & 0 & 0 \\
              0 & 0 & 0 & 0 & 0 & \sqrt{2} & 0 & \sqrt{2} & 0 \\
              0 & 0 & 0 & \sqrt{2} & 0 & 0 & 0 & \sqrt{2} & 0 \\
              \sqrt{2} & 0 & 0 & 0 & \sqrt{2} & 0 & 0 & 0 & \sqrt{2} \\
              0 & \sqrt{2} & 0 & 0 & 0 & \sqrt{2} & 0 & 0 & 0
      \end{pmatrix}.
      \label{eqn:matrix}
 \end{equation*}

\begin{figure}[htb]
  \centering
  \includegraphics[width=\textwidth]{pictures/Würfel.pdf}
  \caption{Würfelüositionen und der schematische Strahlenverlauf für die Messungen.}
  \label{fig:Pos}
\end{figure}