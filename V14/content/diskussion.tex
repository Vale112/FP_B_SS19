\section{Diskussion}
\label{sec:Diskussion}
Insgesamt werden sinnvolle Ergebnisse erhalten.
%Es wird eine deutlich geringere Ausgangsintensität vom Detektor registriert, was auf einen Absorptionsvorgang deutet.
Die Messung zeigt, dass die Würfel 2 und 3 aus homogenen Elementarwürfeln bestehen, da die berechneten Extinktionskoeffizienten (Tabelle \ref{tab:Ex}) nahe beieinander liegen.
Bei der ersten Messung mit dem leeren Würfel fällt auf, dass der Aluminiumrand eines jeden Würfels einen Einfluss auf den Absoprtionskoeffizienten hat, je nach Ausrichtung des Würfels mehr oder weniger.
Es werden die Ergebnisse 
\begin{align*}
    \bar{\mu_1}=0,18\pm0,08\\
    \bar{\mu_2}=0,77\pm0,28\\
    \bar{\mu_3}=0,31\pm0,05
\end{align*}
berechnet.
Ein Vergleich mit den Theoriewerten lässt darauf schließen, dass es sich bei Würfel 2 um Eisen (rel. Abweichung \SI{35}{\percent}) und bei Würfel 3 um Messing (rel. Abweichung \SI{50}{\percent}) handelt.
Der letzte Würfel ist eine Zusammensetzung aus mehreren unterschiedlichen Elementarwürfeln.
Dessen Ergebnisse sind in Tabelle \ref{tab:W5} zusammengefasst.
Schlussendlich kann gesagt werden, dass die Genauigkeit der Messung unter der Aluminiumhülle der Würfel (die zum Schutz von $\beta$-Strahlung dient) und dem ausgedehnten Strahlengang der Quelle gelitten hat.
Der Effekt der Aluminiumhülle kann durch eine Ausgleichsrechnung minimiert werden.
Bei einem ausgedehnten Strahl ist es unvermeidlich, dass nicht alle $\gamma$-Quanten durch den zu untersuchenden Elementarwürfel gehen.
Auch durch die Justierung des Würfels per Hand und Augenmaß verliert die Messung an Präzision.
Somit ist es sinnvoll den Strahlengang der Apparatur zu verringern, um eine Ausdehnung des Strahls vorzubeugen.
Dadurch wird das Justieren des Würfels genauer, weil es per Augenmaß besser zu bestimmen ist wo der Strahlengang verläuft.
Da sowieso nur die mittlere Schicht des 3x3x3 Würfels ausgemessen wurde, hätte das zu untersuchende Objekt auch nur eine 3x3 Schicht des Würfel mit größeren Elementarwürfeln sein können.
Ein Elementarwürfel könnte dadurch vom gesamten Strahl getroffen werden.
