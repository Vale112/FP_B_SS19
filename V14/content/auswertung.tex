\newpage
\section{Auswertung}
\label{sec:Auswertung}
\subsection{Nullmessung}
Für die Bestimmung von der Eingangsintensität $I_0$, wird eine Nullmessung durchgeführt.
Das aufgenommene Spektrum ist in Abbildung \ref{fig:Nullmessung} zu sehen.
Die Intensität wird aus dem Verhältnis der gezählten Counts $C$ pro Zeitintervall $\Delta t$ bestimmt mit
\begin{equation}
    I_0=\frac{C}{\Delta t}.
\end{equation}
Nach Gauß ergibt sich der Fehler zu
\begin{equation}
    \sigma_{I_0}=\frac{\sigma_C}{\Delta t}.
\end{equation}
Es werden $\num{6645\pm104}$ Ereignisse in \SI{60,48}{\second} vom Programm registriert.
Daraus folgt eine Eingangsintensität von $I_0=\SI{110.0\pm1.7}{\becquerel}$.

 \begin{figure}[htb]
   \centering
   \includegraphics[width=\textwidth]{build/Nullmessung.pdf}
   \caption{Nullmessung.}
   \label{fig:Nullmessung}
 \end{figure}
\subsection{Würfelmessungen}
Die Würfelgeometrie wird durch Matrix \eqref{eqn:matrix} beschrieben.
\begin{equation*}
    A = \begin{pmatrix}
               1 & 0 & 0 & 1 & 0 & 0 & 1 & 0 & 0 \\
               0 & 1 & 0 & 0 & 1 & 0 & 0 & 1 & 0 \\
               0 & 0 & 1 & 0 & 0 & 1 & 0 & 0 & 1 \\
               1 & 1 & 1 & 0 & 0 & 0 & 0 & 0 & 0 \\
               0 & 0 & 0 & 1 & 1 & 1 & 0 & 0 & 0 \\
               0 & 0 & 0 & 0 & 0 & 0 & 1 & 1 & 1 \\
               0 & \sqrt{2} & 0 & \sqrt{2} & 0 & 0 & 0 & 0 & 0 \\
               0 & 0 & \sqrt{2} & 0 & \sqrt{2} & 0 & \sqrt{2} & 0 & 0 \\
               0 & 0 & 0 & 0 & 0 & \sqrt{2} & 0 & \sqrt{2} & 0 \\
               0 & 0 & 0 & \sqrt{2} & 0 & 0 & 0 & \sqrt{2} & 0 \\
               \sqrt{2} & 0 & 0 & 0 & \sqrt{2} & 0 & 0 & 0 & \sqrt{2} \\
               0 & \sqrt{2} & 0 & 0 & 0 & \sqrt{2} & 0 & 0 & 0
       \end{pmatrix}.
       \label{eqn:matrix}
  \end{equation*}
Die gemessenen Counts $C$ und errechneten Ausgangsintensitäten $I_j$ des ersten und fünften Würfels sind in Tabelle \ref{tab:W15} einzusehen.
Die Messzeiten sind \SI{30}{\second} für Würfel 1 und \SI{300}{\second} für Würfel 5.
Die entsprechenden Werte für Würfel 2 und 3 stehen in Tabelle \ref{tab:W23} mit den Messzeiten \SI{400}{\second} für Würfel 2 und \SI{250}{\second} für Würfel 3.
Da beide Würfel homogen sind werden äquivalente Projektionen nicht erwähnt.
Es werden die selben Intensitäten für diese Richtungen angenommen.
Der Extinktionskoeffizient $\mu$ wird nach \eqref{eqn:Absorbtionskoeffizenten} berechnet.
Dessen Fehler ergeben sich aus \eqref{eqn:Unsicherheit}.
\begin{table}[H]
    \centering
    \begin{tabular}{|c|c|c|c|c|}  
    \cline{2-5}
    \multicolumn{1}{c|}{} &\multicolumn{2}{c|}{Würfel 1} & \multicolumn{2}{c|}{Würfel 5}\tabularnewline
    \hline
    j & C & $I_{j}\,\left[\text{s}^{-1}\right]$ & C & $I_{j}\,\left[\text{s}^{-1}\right]$\tabularnewline
    \hline
    1 & $2796\pm73$ & $93,2\pm2,4$ & $2812\pm82$& $9,4\pm0,3$\tabularnewline
    \hline
    2 & $1096\pm52$& $36,5\pm1,7$ & $8101\pm136$ & $27,0\pm0,5$\tabularnewline
    \hline
    3 & $3265\pm76$ & $108,8\pm2,5$ & $7769\pm133$ & $25,9\pm0,4$\tabularnewline
    \hline
    4 & $1897\pm64$ & $63,2\pm2,1$ & $4391\pm112$ & $14,6\pm0,4$\tabularnewline
    \hline
    5 & $1368\pm59$ & $45,6\pm2,0$ & $7408\pm132$ & $24,7\pm0,4$\tabularnewline
    \hline
    6 & $1985\pm69$ & $66,2\pm2,3$ & $4500\pm96$ & $15,0\pm0,3$\tabularnewline
    \hline
    7 & $1735\pm63$ & $57,8\pm2,1$ & $6279\pm121$ & $20,9\pm0,4$\tabularnewline
    \hline
    8 & $1046\pm53$ & $34,8\pm1,8$ & $4480\pm105$ & $14,9\pm0,4$\tabularnewline
    \hline
    9 & $1920\pm67$ & $64,0\pm2,2$ & $4855\pm113$ & $16,2\pm0,4$\tabularnewline
    \hline
    10 & $1973\pm70$ & $65,8\pm2,3$ & $13795\pm170$ & $46,0\pm0,6$\tabularnewline
    \hline
    11 & $1833\pm93$ & $61,1\pm3,1$ & $3839\pm92$ & $12,8\pm0,3$\tabularnewline
    \hline
    12 & $1474\pm59$ & $49,1\pm2,0$ & $3961\pm95$ & $13,2\pm0,3$\tabularnewline
    \hline
    \end{tabular}
    \caption{Messwerte der Würfel 1 und 5.}
    \label{tab:W15}
    \end{table}


\begin{table}[H]
        \centering
        \begin{tabular}{|c|c|c|c|c|}
        \cline{2-5}
        \multicolumn{1}{c|}{} &\multicolumn{2}{c|}{Würfel 2} & \multicolumn{2}{c|}{Würfel 3}\tabularnewline
        \hline
        j & C & $I_{j}\,\left[\text{s}^{-1}\right]$ & C & $I_{j}\,\left[\text{s}^{-1}\right]$\tabularnewline
        \hline
        1 & $3766\pm95$ & $9,4\pm0,2$  & $9656\pm154$ & $38,6\pm0,6$\tabularnewline
        \hline
        2 & $3799\pm94$& $9,5\pm0,2$  & $9113\pm153$ & $36,5\pm0,6$\tabularnewline
        \hline
        3 & $3766\pm95$ & $9,4\pm0,2$  & $9656\pm154$ & $38,6\pm0,6$\tabularnewline
        \hline       
        10 & $1973\pm70$ & $11,3\pm2,8$   & $12011\pm170$ & $48,0\pm0,7$\tabularnewline
        \hline
        11 & $1833\pm93$ & $7,2\pm0,2$  & $9366\pm147$ & $37,5\pm0,6$\tabularnewline
        \hline
        12 & $1973\pm70$ & $11,3\pm2,8$  & $12011\pm170$ & $48,0\pm0,7$\tabularnewline
        \hline
        \end{tabular}
        
        \caption{Messwerte der Würfel 2 und 3.}
        \label{tab:W23} 
        \end{table}

\FloatBarrier
Aus der Berechnung mit den Daten aus den vorherigen Tabellen folgen die Extinktionskoeffizienten in Tabelle \ref{tab:Ex}.
Die Würfel 1,2 und 3 bestehen alle aus homogenen Elementarwürfeln (Würfel 1 ist komplett leer).
Daher ist es sinnvoll einen Mittelwert anzugeben. 
Der Mittelwert errechnet sich nach 
\begin{equation*}
    \bar{\mu}=\frac{\sum_{j=1}^9\mu_j}{9}
\end{equation*}
und dessen Fehler nach 
\begin{equation*}
    \bar{\sigma_\mu}=\frac{\sum_{j=1}^9\sigma_{\mu_j}}{9}.
\end{equation*}
Es wird 
\begin{align*}
    \bar{\mu_1}=0,18\pm0,08\\
    \bar{\mu_2}=0,77\pm0,28\\
    \bar{\mu_3}=0,31\pm0,05
\end{align*}
berechnet.
Ein Vergleich mit den Theoriewerten (siehe Tabelle \ref{tab:Theo}) lässt darauf schließen, dass es sich bei Würfel 2 um Eisen (rel. Abweichung \SI{35}{\percent}) und bei Würfel 3 um Messing (rel. Abweichung \SI{50}{\percent}) handelt.
Die Extinktionskoeffizienten für den inhomogenen Würfel 5 stehen in Tabelle \ref{tab:W5}.
Dazu wird dieser noch mit dem Material dessen theoretischer Extinktionskoeffizienten am nähesten liegt verglichen und eine relative Abweichung bestimmt.

\begin{table}[H]
    \centering
    \begin{tabular}{|c|c|c|c|c|}  
    \cline{2-5}
    \multicolumn{1}{c|}{} & \multicolumn{4}{c|}{$\mu_j/\si{\per\centi\meter}$} \tabularnewline
    \hline
    \multicolumn{1}{|c|}{j} &\multicolumn{1}{c|}{Würfel 1} & \multicolumn{1}{c|}{Würfel 2} & \multicolumn{1}{c|}{Würfel 3} & \multicolumn{1}{c|}{Würfel 5}\tabularnewline
    \hline
    1 & $-0,03\pm0,09$ & $0,75\pm0,33$ & $0,31\pm0,05$& $1,00\pm0,06$\tabularnewline
    \hline
    2 & $0,40\pm0,07$& $0,85\pm0,24$ & $0,33\pm0,04$ & $0,77\pm0,65$\tabularnewline
    \hline
    3 & $0,06\pm0,09$ & $0,75\pm0,33$ & $0,31\pm0,05$ & $0,18\pm0,06$\tabularnewline
    \hline
    4 & $0,08\pm0,07$ & $0,85\pm0,24$ & $0,33\pm0,04$ & $0,034\pm0,4$\tabularnewline
    \hline
    5 & $0,54\pm0,07$ & $0,54\pm0,25$ & $0,24\pm0,04$ & $0,17\pm0,05$\tabularnewline
    \hline
    6 & $0,15\pm0,07$ & $0,85\pm0,24$ & $0,33\pm0,04$ & $0,87\pm0,05$\tabularnewline
    \hline
    7 & $0,22\pm0,09$ & $0,75\pm0,33$ & $0,31\pm0,05$ & $1,11\pm0,06$\tabularnewline
    \hline
    8 & $0,26\pm0,07$ & $0,85\pm0,24$ & $0,33\pm0,04$ & $0,41\pm0,05$\tabularnewline
    \hline
    9 & $-0,09\pm0,09$ & $0,75\pm0,33$ & $0,31\pm0,05$ & $0,40\pm0,06$\tabularnewline
    \hline
    \end{tabular}
    
    \caption{Extinktionskoeffizienten der Würfel.}
    \label{tab:Ex}
    \end{table}
\FloatBarrier



\begin{table}[H]
    \centering
    \begin{tabular}{|c|c|c|c|c|}  
    \hline
    j & {$\mu_j/\si{\per\centi\meter}$} & {Material} & {$\mu_{theo}/\si{\per\centi\meter}$} & {rel. Abw.}\tabularnewline
    \hline
    1 & $1,00\pm0,06$ & {Blei} & 1,25 & {\SI{20}{\percent}}\tabularnewline
    \hline
    2 & $0,77\pm0,05$ & {Messing} & 0,62 & {\SI{24}{\percent}}\tabularnewline
    \hline
    3 & $0,18\pm0,06$ & {Aluminium} & 0,20 & {\SI{10}{\percent}}\tabularnewline
    \hline
    4 & $0,34\pm0,05$ & {Eisen} & 0,57 & {\SI{40}{\percent}}\tabularnewline
    \hline
    5 & $0,17\pm0,05$ & {Aluminium} & 0,20 & {\SI{15}{\percent}}\tabularnewline
    \hline
    6 & $0,87\pm0,05$ & {Messing} & 0,62 & {\SI{40}{\percent}}\tabularnewline
    \hline
    7 & $1,11\pm0,06$ & {Blei} & 1,25 & {\SI{11}{\percent}}\tabularnewline
    \hline
    8 & $0,41\pm0,05$ & {Eisen} & 0,57 & {\SI{28}{\percent}}\tabularnewline
    \hline
    9 & $0,40\pm0,06$ & {Eisen} & 0,57 & {\SI{30}{\percent}}\tabularnewline
    \hline
    \end{tabular}
    
    \caption{Extinktionskoeffizienten des Würfels 5.}
    \label{tab:W5}
    \end{table}
\FloatBarrier

\begin{table}[H]
    \centering
    \begin{tabular}{|c|c|c|c|}
    \hline 
    Material & $\sigma\,\left[\frac{\text{cm}^{2}}{\text{g}}\right]$ & $\rho\,\left[\frac{\text{g}}{\text{cm}^{3}}\right]$ & $\mu\,\left[\text{cm}^{-1}\right]$\tabularnewline
    \hline 
    \hline 
    Aluminium & 0,075 & 2,70 & 0,20\tabularnewline
    \hline 
    Blei & 0,110 & 11,34 & 1,25\tabularnewline
    \hline 
    Eisen & 0,074 & 7,87 & 0,57\tabularnewline
    \hline 
    Messing & 0,073 & 8,45 & 0,62\tabularnewline
    \hline 
    Delrin & 0,081 & 1,42 & 0,11\tabularnewline
    \hline 
    \end{tabular}
    \caption{Literaturwerte der Stoffdichte, Massenschwächungskoeffizient ($E_{\gamma}=\SI{661}{\kilo\electronvolt}$) und der
    Absorptionskoeffizienten \cite{Theorie}}
    \label{tab:Theo}
    \end{table}