\section{Diskussion}
\label{sec:Diskussion}
Nach dem Abzug des ermittelten Untergrunds werden einige Werte kleiner als Null, sodass in den \eqref{eqn:int} und \eqref{eqn:lin2} Beträge
verwendet wurden. Ebenso wurde erst nach Abzug der verwendete Bereich und das Maximum festgelegt.

Für die ermittelten Aktivierungsenergien im Abschnitt \ref{sec:klT} ergibt sich für die Heizrate von etwa \SI{1}{\per\kelvin} eine 
Abweichung von \SI{7}{\percent} zum Literaturwert $W=\SI{0.66(1)}{\electronvolt} = \SI{1.06(2)e-19}{\joule}$ \cite{quelle}.
Für die Heizrate von  etwa \SI{2}{\per\kelvin} eine Abweichung von \SI{45}{\percent}. 
Die Abweichungen für die Aktivierungsenergien, welche in Abschnitt \ref{sec:grT} über den Polarisationsansatz ermittelt werden sind mit
\SI{29}{\percent} für die Heizrate von etwa \SI{1}{\per\kelvin} und \SI{46}{\percent} für die andere, größer. Dies 
widerspricht den Erwartungen, hier eine genauere Messung zu erhalten. Ein Grund für die großen Abweichungen
bei der Intgralmethode hat ihren Ursprung wohl in der nummerischen Berechnung der Werte.

Da die Abweichung der Relaxationszeit $\tau_0$ vom Literaturwert $\tau_0 = \SI{4(2)e-14}{\second}$ um mehrere Größenordnungen
abweicht, wird eine Errechnung der Abweichung als nicht sinnvoll betrachtet.
Diese Abweichungen können zum teil damit begründet werden, dass die Probe hydroskopisch ist 
und somit Wasser aufnimmt. Dies führt nicht nur zu einer Veränderung der Struktur als solches, sondern da Wasser selbst ein 
Dipol ist zu einer deutlichen Verfäschung der Messergebnisse. Die Ursache darin liegt, dass die Probe sich in keinem 
Ultrahochvakuum befand, sondern nur in einem Vorvakuum von etwa \SI{19}{\milli\bar}. 
Weitere Gründe sind die geringen fließenden Ströme, welche stark auf elektrische und magnetische Feldänderungen reagieren, sowie
die nicht konstante Heizrate, welche besonders für die Soll-Heizrate von \SI{2}{\kelvin\per\minute} starke Abweichungen zeigte, dies 
spiegelt sich auch in den Fehlern wieder, da hier die Abweichungen größer als für die kleinere Heizrate sind.


Das höhere erste Maximum in Abbildung \ref{fig:Messdaten2} im Bezug zum ersten Maximum in Abbildung \ref{fig:Messdaten1}, lässt
sich dadurch erklären, dass für einen schnelleren Temeperaturanstieg mehr Dipole gleichzeitig relaxaxieren. Denn die Relaxationszeit
ist nach Formel \eqref{ref:Relaxzeit} bekanntlich exponentiell von der reziproken Temperatur abhängig.
Die Position des Maximums verschiebt sich des Weiteren für höhere Heizraten zu höheren Temperaturen, da der Temeperaturanstieg
schneller abläuft, als dass die Dipole direkt folgen könnten und sich wieder statistisch verteilen.

