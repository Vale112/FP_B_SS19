\newpage
\section{Auswertung}
\subsection{Erdmagnetfeld und Landé-Faktoren}
\label{sec:Auswertung}
% \subsection{Unterkapiel}
% \label{sec:Unterkapitel}

% \begin{figure}
%   \centering
%   \includegraphics[width=\textwidth]{Plot.pdf}
%   \caption{Bildunterschrift}
%   \label{fig:Plot1}
% \end{figure}
\FloatBarrier
Die Magnetfelder werden aus der Formel
\begin{equation}
B=\mu_0\frac{8 I N}{\sqrt{125}R}
\end{equation}
für Helmholtzspulenpaare errechnet.
Hierbei ist $I$ die abgelesene Stromstärke, $N$ die Anzahl der Windungen und $R$ der Radius der Spulen.

Zunächst wird die vertikale Komponente $B_{\text{vert}}$ des Erdmagnetfeldes kompensiert.
Das dafür angelegte Magnetfeld beträgt
\begin{equation}
B_{\text{vert}}=\SI{3.448e-5}{\tesla},
\end{equation}
wobei der aufgenommene Peak minimal wird.
Ein typisches Signalbild ist in Abbildung \ref{fig:Signal} zu sehen.
 \begin{figure}
   \centering
   \includegraphics[width=10cm]{pictures/TEK0008.JPG}
   \caption{Typisches Signal am Oszilloskop.}
   \label{fig:Signal}
 \end{figure}
 \FloatBarrier

Die Resonanzpositionen sind in Tabelle \ref{tab:RF} einzusehen und in Abbildung \ref{fig:B} zusätzlich zu einem linearen Fit graphisch dargestellt.
Die lineare Regression hat die Form
\begin{equation*}
B=a\nu_{\text{RF}}+b.
\end{equation*}
Ein Blick auf Formel \eqref{eqn:B} verrät, dass 
\begin{equation*}
a=\frac{h}{\mu_{\text{B}}g_{\text{F}}}
\end{equation*}
und damit 
\begin{equation*}
g_{\text{F}}=\frac{h}{\mu_{\text{B}}a}
\end{equation*}
gilt.
Für die ersten Resonanzstellen ergibt sich
\begin{align*}
a&=\SI{142.1(9)e-6}{\tesla\per\mega\hertz}\\
b&=\SI{19.9(5)e-6}{\tesla} \\
\Rightarrow g_{\text{F,1}}&=\num(0.502(3)}
\end{align*}
und aus den zweiten
\begin{align*}
a&=\SI{212(1)e-6}{\tesla\per\mega\hertz}\\
b&=\SI{20.2(6)e-6}{\tesla}\\
\Rightarrow g_{\text{F,2}}&=\num(0.336(1)}.
\end{align*}
Somit ergibt sich das Verhältnis der beiden Faktoren zu
\begin{equation*}
\frac{g_{\text{F,1}}}{g_{\text{F,2}}}=\num{1.49(1)}.
\end{equation*}
Die horizontale Komponente des Erdmagnetfeldes lässt sich aus dem Fitparameter $b$ auslesen.
Der Mittelwert der berechneten Werte beträgt
\begin{equation*}
 B_{\text{hor}}=\SI{2.01(4)e-5}{\tesla}.
\end{equation*}
\begin{table}[h]
  \centering
  \begin{tabular}{S[table-format=1.1] S[table-format=3.2] S[table-format=3.2]}
    {$\nu_\text{RF}$\;/\;\si{\mega\hertz}} & {$B_1$\;/\;\si{\micro\tesla}} & {$B_2$\;/\;\si{\micro\tesla}} \\
    \midrule
    0.1 &  33.98&   41.28 \\
    0.2 &  48.93 &   63.35 \\
    0.3 &  62.62 &   83.99 \\
    0.4 &  76.17 &  104.59 \\
    0.5 &  90.02 &  125.44 \\
    0.6 & 105.82 &  149.33 \\
    0.7 & 120.44 &  170.10 \\
    0.8 & 134.89 &  190.48 \\
    0.9 & 147.53 &  210.25 \\
    1.0 & 161.26 &  233.12
  \end{tabular}
  \caption{Resonanzpositionen abhängig von der RF-Frequenz.}
  \label{tab:RF}
\end{table}
 \begin{figure}
   \centering
   \includegraphics[width=\textwidth]{build/B.pdf}
   \caption{Frequenz gegen B-Feldstärke der Resonanzstellen aufgetragen.}
   \label{fig:B}
 \end{figure}
 \FloatBarrier
 \subsection{Kernspins}
 Die Quantenzahlen für Rubidium betragen
 \begin{equation*}
 L = 0,\quad S = \frac{1}{2},\quad J = \frac{1}{2}
 \end{equation*}
 wodurch mit Hilfe von Gleichung \eqref{eqn:gj} sich der Faktor
 \begin{equation*}
 g_{\text{J}}=\num{2.0023}
 \end{equation*}
 ergibt.
 Einsetzen von $F=I+J$ in \eqref{eqn:lande} und umformen nach dem Kernspin $I$ ergibt
 \begin{equation}
 I=\frac{g_J-4g_F}{4g_F} + \sqrt{\left(\frac{g_J-4g_F}{4g_F}\right)^2-\frac{3}{4}\left(1-\frac{g_J}{g_F}\right)}.
 \end{equation}
 Mit den vorher berechneten Werten der Landéschen Faktoren ergibt sich
 \begin{align*}
 I_1 &= \num{1.49(1)}\\
 I_2 &= \num{2.57(1)} 
 \end{align*}
 für die Kernspins.
 Der Vergleich mit der Literatur \cite{nudat2}
\begin{align*}
{}^{87}\text{Rb}:\quad I&=\frac{3}{2}\\
\\
{}^{85}\text{Rb}:\quad I&=\frac{5}{2}
\end{align*}
verrät, dass die erste Resonanz zum Isotop $\ce{^{87}Rb}}$ gehört und die zweite zum Isotop $\ce{^{85}Rb}}$.
\subsection{Isotopenverhältnis}
Aus Abbildung \ref{fig:Signal} wird das Verhältnis der Amplituden zueinander ausgemessen.
Da die genaue Messung mit dem Oszilloskop nicht erfolgt ist, wird versucht die Amplitudentiefe per Gimp \cite{gimp} zu erfassen.
Die Höhen der Amplituden entsprechen
\begin{align*}
h_1&=\SI{92.0(9)}{\px}\\
h_2&=\SI{192(2)}{\px}
\end{align*}
mit einem Messfehler von \SI{1}{\percent}.
Damit ergibt sich ein Verhältnis von
\begin{equation}
  \frac{h_1}{h_2}=\num{0.479(6)}.
\end{equation}
Aus der Literatur \cite{nudat2} wird das Verhältnis zu 
\begin{equation}
  \frac{h_1}{h_2}=\num{0.39}
\end{equation}
bestimmt.
\subsection{Quadratischer Zeeman-Effekt}
Die Effekte des quadratischen Zeemaneffekts werden aus Gleichung \eqref{eqn:zeemanquadrat} hergeleitet.
Die für die Rechnung benutzten Werte sind in Tabelle \ref{tab:qZ} angegeben.
\begin{align*}
{}^{87}\text{Rb}: U_{\text{Z}}&=\SI{7.52(5)e-28}{\joule}\\
{}^{85}\text{Rb}: U_{\text{Z}}&=\SI{7.27(4)e-28}{\joule}
\end{align*}
\begin{table}
  \centering
  \caption{Werte zur Bestimmung des quadratischen Zeemaneffekts.}
  \label{tab:qZ}
  \begin{tabular}{c c S[table-format=1.2e2] S[table-format=1.4] @{${}\pm{}$} S[table-format=1.4] c[table-format=3.0]}
    Isotop & $M_F$ & {$\symup{\Delta}E_\text{Hy}$\;/\;\si{\joule}} & \multicolumn{2}{c}{$g_F$} & {$B$\;/\;\si{\micro\tesla}} \\
    \midrule
    ${}^{85}$Rb & 0 & 2.01e-24 & 0.3361 & 0.0017 & 233.12 \\
    ${}^{87}$Rb & 0 & 4.53e-24 & 0.5026 & 0.0032 & 161.26 \\
  \end{tabular}
\end{table}