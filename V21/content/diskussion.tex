\section{Diskussion}
\label{sec:Diskussion}
Das eingestellte Magnetfeld zur Kompensation der vertikalen Komponente des Erdmagnetfeldes beträgt \SI{34.48}{\micro\tesla}.
Das entspricht einer Abweichung von \SI{31}{\percent} zum Literaturwert \cite{noaa} von \SI{45.19}{\micro\tesla}.
Das eingestelle Horizontalwert mit $B_{\text{hor}}=\SI{2.01(4)e-5}{\tesla}$ weicht nur um \SI{4}{\percent} vom Literaturwert \SI{19.3}{\micro\tesla} ab.
Für so geringe Werte zeichnen diese kleinen Abweichungen eine gute Messung aus.

Die Landéfaktoren müssen laut Literatur \cite{wang} ein Verhältnis von $\num{1.49(1)}$ aufzeigen.
Dieses Ergebnis wurde perfekt nachgestellt.

Auch die Messung der Kernspins zeigt eine sehr gute Messung.
Die höhere Abweichung zwischen dem gemessenen Kernspin $I_2=\num{2.57(1)}$ und dem Literaturwert von $I=\num{2.5}$ beträgt nur \SI{3}{\percent}.

Lediglich das Isotopenverhältnis zeigt eine größere Abweichung der Messwerte vom Literaturwert von \SI{23}{\percent}.
Die Ursache mag darin liegen, dass die Amplitude nicht mit dem Oszilloskop gemessen wurde, sondern deren Längen in Pixel via Gimp bestimmt wurden.