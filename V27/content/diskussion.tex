\section{Diskussion}
\label{sec:Diskussion}
\begin{table}
  \caption{Zusammenfassung der Ergebnisse.}
  \label{tab:results}
  \begin{tabular}{c S[table-format=1.2] S[table-format=1.2] @{${}\pm{}$} S[table-format=1.2] S[table-format=2.0]}
    \toprule
    {Aufspaltung} & {$g_{\text{theo}}$} & \multicolumn{2}{c}{$g_{\text{exp}}$} & {mittlere Abweichung / \%} \\
    \midrule
    rot  $\sigma$  & 1 & 0.99   & 0.11 & 1  \\
    blau $\sigma$ & 1.75 & 1.82   & 0.13  & 4    \\
    blau $\pi$    & 0.5 & 0.43 & 0.16 & 14 \\
    \bottomrule
  \end{tabular}
\end{table}
Die zusammengefassten Werte aus diesem Experiment sind nochmal in Tabelle \ref{tab:results} zu sehen.
Insgesamt wurden gute Ergebnisse erzielt, da es nur zu geringen Abweichungen kommt.
Die geringen Fehler bei der Kalibrierung des des Magnetfeldes weisen auf eine gute Approximation hin.
Trotzdem sind Ungenauigkeiten durch die unscharfen Bilder und dadurch bei der Datenverarbeitung der aufgenommenen Bilder zustande gekommen.
Die in der Tabelle stehenden "Übergangslandéfaktoren" sind eine Mittelung aller Landefaktoren eines möglichen Übergangs mit gleicher Magnetquantenzahl $m_j$,
da es experimentell nicht möglich war diese auseinander zu halten.
Der $\pi-$Übergang der roten Linie mit $g=0$ für alle Übergänge konnte experimentell nicht ermittelt werden, da keine Aufspaltung der Spektrallinie im Magnetfeld stattfindet.