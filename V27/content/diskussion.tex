\section{Diskussion}
\label{sec:Diskussion}
\begin{table}
  \caption{Zusammenfassung der Ergebnisse.}
  \label{tab:results}
  \begin{tabular}{l S S[table-format=1.3]@{${}\pm{}$} S[table-format=1.3] S}
    \toprule
    {Aufspaltung} & {$g_{ij}$ (theoretisch)} & \multicolumn{2}{c}{$g_{ij}$ (experimentell)} & {mittlere Abweichung / \%} \\
    \midrule
    rot  $\sigma$  & 1 & 0,99   & 0,11 & 1  \\
    blau $\sigma$ & 1,75 & 1,82   & 0,13  & 4    \\
    blau $\pi$    & 0,5 & 0,43 & 0,16 & 14 \\
    \bottomrule
  \end{tabular}
\end{table}
Die Werte aus dem Experiment sind in Tabelle \ref{tab:results} zu sehen.
Insgesamt wurden gute Ergebnise erzielt.
Die geringen Fehler geben Ausschluss auf eine korrekte Eichung des Magnetfeldes.
Trotzdem sind Ungenauigkeiten durch die unscharfen Bilder und eventuellen Fehlern bei der Datenverarbeitung der aufgenommenen Bilder zustande gekommen.
Die in der Tabelle stehenden "Übergangslandefaktoren" sind eine Mittelung aller Landefaktoren eines möglichen Übergangs mit gleicher Magnetquantenzahl $m_j$, da es experimentell nicht möglich war diese auseinander zu halten.
Der $\pi-$Übergang der roten Linie mit $g_{i,j}$ konnte experimentell nicht ermittelt werden, da keine Aufspaltung der Spektrallinie im Magnetfeld stattfindet.