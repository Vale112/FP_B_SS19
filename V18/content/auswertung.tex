\newpage
\section{Auswertung}
\label{sec:Auswertung}
Die gesamte Auswertung wird mit Hilfe der Bibilotheken \cite{matplotlib}, \cite{numpy}, \cite{scipy} und
\cite{uncertainties} in \textit{python} durchgeführt. Bei den abgebildeten Spektren handelt es sich stets um 
Ausschnitte der gesamt Spektren, welche \num{8192} Kanäle besitzen. Es ist stets nur der markante Bereich
mit den relevanten Peaks dargestellt.


\subsection{Energiekalibration}
\label{sec:Energiekalibration}
Zu Beginn des Versuchs wird zunächst die Probe ${}^{152}$Eu untersucht. Die Messwerte
sind in Abbildung \ref{fig:Eu_log_Kanal} logarithmisch dargestellt.
Die für die Kalibrierung verwendeten Peaks sind mit roten Kreuzen makiert. Die Messwerte
werden mit den theoretischen Werten \cite{referenz1} verglichen, sodass mit Hilfe einer linearen Ausgleichsrechnung
eine Zuordnung von Kanalnummer $i$ und Energie $E$ möglich ist. Diese Daten sind
gemeinsam mit den Emissionswahrscheinlichkeit in Tabelle \ref{tab:zuordnung_eu}
gelistet, dabei werden Energien mit einer Emissionswahrscheinlichkeit $W$
kleiner als \SI{2}{\percent} vernachlässigt.
Da die Peaks einen Gauß-Glocken förmigen Verlauf haben, wird zunächst ein Gauß-Fit der Form
\begin{equation}
  N(i) = h \cdot \exp{\left(-\frac{\left(i - \mu\right)^2}{\sigma^2}\right)} + a
\end{equation}
über die identifizierten Peaks durchgeführt. Die Ergebnisse dieses Fits sind in Tabelle \ref{tab:gauss_parameter}
dargestellt, dabei bezeichnet $\mu$ den Mittelwert, $h$ die Höhe des Peaks, $\sigma$ die Standardabweichung und 
$a$ ein Parameter, welcher den Untergrund darstellt. Für den Fit wird der detektierte Kanal und die auf jeder Seite 
\num{40} nebenliegenden Kanäle verwendet. Anschließend wird mit der gefitteten Kanalnummer die Ausgleichsrechnung
durchgeführt. Für die Ausgleichsrechnung wird eine Gerade der Form
\begin{equation}
  E = m \cdot i + b
\end{equation}
verwendet, wobei $m$ die Steigung und $b$ der y-Achsenabschnitt ist. Dabei lassen sich die Parameter zu
\begin{gather*}
  m = \SI{0.40302(2)}{\kilo\electronvolt\per\kanal} \\
  b = \SI{-2.63(4)}{\kilo\electronvolt}
\end{gather*}
bestimmen.
In der Abbildung \ref{fig:kalibration} ist das Ergebnis der Ausgleichsrechnung graphisch
dargestellt, dabei befindet sich auf der x-Achse die Kanalnummer $i$ und auf der
y-Achse die errechnete Energie $E_\text{i}$.


\begin{figure}[htb]
 \centering
 \includegraphics[width=\textwidth]{Eu_log_Kanal.pdf}
 \caption{Die Zählrate $N$ in Abhängigkeit der Kanalnummer $i$ für ${}^{152}$Eu mit logarithmierter y-Skala.}
 \label{fig:Eu_log_Kanal}
\end{figure}

\begin{figure}[htb]
 \centering
 \includegraphics[width=\textwidth]{build/kalibration.pdf}
 \caption{Die ermittelte Energie anhand der Daten in Abhängigkeit der Kanalnummer zur Kalibartion des Ge-Detektors.}
 \label{fig:kalibration}
\end{figure}
\begin{table}
	\centering
	\caption{Gegebene Werte zur Kalibrierung des Germanium-Detektors \cite{referenz1}.}
	\label{tab:zuordnung_eu}
	\begin{tabular}{
		S[table-format=4.4] @{${}\pm{}$} S[table-format=1.4]
		S[table-format=2.3] @{${}\pm{}$} S[table-format=1.3]
		S[table-format=4.0]
		}
	\toprule
		\multicolumn{2}{c}{ $E_\text{i}$\;/\;\si{\kilo\electronvolt}} &
		\multicolumn{2}{c}{ $W_\text{i}$\;/\;\si{\percent}} &
		{$i$} \\
	\midrule
		 121.7817 &  0.0003 &  28.41 &  0.13 &  309 \\
		 244.6974 &  0.0008 &  7.55 &  0.04 &  614 \\
		 295.9387 &  0.0017 &  0.442 &  0.003 &  740 \\
		 344.2785 &  0.0012 &  26.59 &  0.12 &  861 \\
		 411.1165 &  0.0012 &  2.238 &  0.010 &  1027 \\
		 443.965 &  0.003 &  3.120 &  0.028 &  1108 \\
		 778.9045 &  0.0024 &  12.97 &  0.06 &  1939 \\
		 867.380 &  0.003 &  4.243 &  0.023 &  2159 \\
		 964.079 &  0.018 &  14.50 &  0.06 &  2399 \\
		 1085.837 &  0.010 &  10.13 &  0.06 &  2702 \\
		 1112.076 &  0.003 &  13.41 &  0.06 &  2765 \\
		 1408.013 &  0.003 &  20.85 &  0.08 &  3500 \\
	\bottomrule
	\end{tabular}
\end{table}
\begin{table}
	\centering
	\caption{Parameter des durchgeführten Gauss-Fits pro Kanal für das Spektrum von ${}^{152}$Eu.}
	\label{tab:gauss_parameter}
	\begin{tabular}{
		S[table-format=4.0]
		S[table-format=4.3] @{${}\pm{}$} S[table-format=1.3]
		S[table-format=2.1] @{${}\pm{}$} S[table-format=1.1]
		S[table-format=4.0] @{${}\pm{}$} S[table-format=2.0]
		S[table-format=1.3] @{${}\pm{}$} S[table-format=1.3]
		}
	\toprule
		{$i$} &
		\multicolumn{2}{c}{$\mu_\text{i}$} &
		\multicolumn{2}{c}{$a_\text{i}$} &
		\multicolumn{2}{c}{$h_\text{i}$} &
		\multicolumn{2}{c}{$\sigma_\text{i}$} \\
	\midrule
		 309 &  308.773 &  0.005 &  99 &  3 &  5344 &  20 &  1.644 &  0.007 \\
		 614 &  613.758 &  0.015 &  44.6 &  1.1 &  757 &  7 &  1.94 &  0.02 \\
		 740 &  740.872 &  0.187 &  31.4 &  0.7 &   43 &  5 &  1.97 &  0.27 \\
		 861 &  860.839 &  0.007 &  24.8 &  1.0 &  1668 &  6 &  2.167 &  0.009 \\
		 1027 &  1026.680 &  0.054 &  20.1 &  0.6 &  118 &  3 &  2.28 &  0.08 \\
		 1108 &  1108.071 &  0.042 &  18.9 &  0.5 &  134 &  3 &  2.38 &  0.06 \\
		 1939 &  1939.059 &  0.046 &  15.2 &  0.7 &  202 &  3 &  3.35 &  0.07 \\
		 2159 &  2158.525 &  0.137 &  14.4 &  0.5 &   51 &  2 &  3.8 &  0.2 \\
		 2399 &  2398.434 &  0.046 &  8.1 &  0.5 &  160 &  2 &  4.21 &  0.07 \\
		 2702 &  2701.024 &  0.102 &  7.1 &  0.7 &  101 &  3 &  4.71 &  0.15 \\
		 2765 &  2765.719 &  0.080 &  6.2 &  0.6 &  115 &  2 &  4.83 &  0.12 \\
		 3500 &  3500.536 &  0.093 &  1.1 &  0.7 &  119 &  3 &  5.50 &  0.14 \\
	\bottomrule
	\end{tabular}
\end{table}
\FloatBarrier

\subsection{Vollenergienachweiseffizienz}
\label{sec:Vollenergienachweiseffizenz}
Die Aktivität der Probe am $15.04.2019$, dem Versuchstag lässt sich mit
\begin{equation}
  A = A_0 \cdot \exp{\left(-\lambda \cdot t\right)}
\end{equation}
berechnen, hierbei ist $A_0$ die Aktivität am Produktionstag,
$\lambda = \SI[per-mode = reciprocal-positive-first]{1.6244(19)e-9}{\per\second}$ \cite{referenz1} 
die Zerfallskonstante und $t$ die Zeit. Es ist aus \cite{V18} bekannt, dass die Probe am Produktionstag, dem
$01.10.2000$ eine Aktivität von $\SI{4130(60)}{\becquerel}$ besaß.
Die Aktivität am Versuchstag beläuft sich damit auf
\begin{equation*}
  \SI{1597(23)}{\becquerel}.
\end{equation*}
Des Weiteren lässt sich mit Hilfe der Formel \ref{eqn:raum} der Raumwinkel errechnen.
Dabei beträgt $a~=~\SI{8.8(1)}{\centi\meter}$, was sich aus dem Abstand der Probe zum
Aluminiumgehäuse (\SI{7.3(1)}{\centi\meter}) und dem Abstand des Gehäuses zum Detektor 
(\SI{1.5}{\centi\meter}) ergibt. Der Radius $r$ des Detektors betrug \SI{2.25}{\centi\meter}, was sich
aus der Abbildung \ref{fig:Detektor} ablesen lässt. Damit lässt sich der Raumwinkel zu
\begin{equation*}
  \frac{\Omega}{4\pi} = \num{0.0156(3)}
\end{equation*}
bestimmen.
Unter Zuhilfenahme von
\begin{equation}
  Z_\text{i} = \sqrt{2\pi} h_\text{i} \sigma_{i}
\end{equation}
lässt sich der Peakinhalt $Z_\text{i}$ bestimmen. Nun lässt sich mit Formel \ref{eqn:eff}
die Nachweiswahrscheinlichkeit des Detektors errechnen. Dabei ist die Messzeit von $t_\text{mess} = \SI{4676}{\second}$
zu berücksichtigen. Die sich ergeben Werte sind in Tabelle \ref{tab:det_eff} gelistet.
Anhand der Werte aus Tabelle \ref{tab:det_eff} lässt sich ein Fit der Form
\begin{equation}
  \label{eqn:Potenz}
  Q(E) = a \cdot \left(\frac{E-b}{\SI{1}{\kilo\electronvolt}}\right)^c + d
\end{equation}
durchführen. Dabei werden nur Peaks der Energie $E_\text{i} > \SI{150}{\kilo\electronvolt}$ berücksichtigt,
da nur hier gegebenen ist, dass sie die Aluminiumhaube und die äußere Schicht des Detektors nahezu
ungehindert durchdringen. Die Parameter ergeben sich dabei zu
\begin{align*}
  a &= \num{-0.01(11)} \\
  b &= \SI{121.81(5)}{\kilo\electronvolt} \\
  c &= \num{0.4(8)} \\
  d &= \num{0.4(5)}.
\end{align*}
Eine graphische Darstellung dieser Ausgleichsrechnung ist in Abbildung \ref{fig:effizenz} zu finden.

\begin{figure}[htb]
 \centering
 \includegraphics[width=\textwidth]{build/efficiency.pdf}
 \caption{Die Vollenergienachweiseffizienz des Detektors gegen die Energie aufgetragen.}
 \label{fig:effizenz}
\end{figure}
\begin{table}
	\centering
	\caption{Die Werte zur Bestimmung der Detektoreffizenz als Ergebnis des Gaußfits.}
	\label{tab:det_eff}
	\begin{tabular}{
		S[table-format=4.2] @{${}\pm{}$} S[table-format=1.2]
		S[table-format=2.3] @{${}\pm{}$} S[table-format=1.3]
		S[table-format=5.0] @{${}\pm{}$} S[table-format=3.0]
		S[table-format=1.3] @{${}\pm{}$} S[table-format=1.3]
		}
	\toprule
		\multicolumn{2}{c}{$E_\text{i}$\;/\;\si{\kilo\electronvolt}} &
		\multicolumn{2}{c}{$W_\text{i}$\;/\;\si{\percent}} &
		\multicolumn{2}{c}{$Z_\text{i}$\;/\;\si{\kilo\electronvolt}} &
		\multicolumn{2}{c}{$Q_\text{i}$\;/\;\si{\becquerel }} \\
	\midrule
		 121.81 &  0.04 &  28.41 &  0.13 &  22027 &  124 &  0.666 &  0.018 \\
		 244.73 &  0.05 &  7.55 &  0.04 &  3685 &  55 &  0.419 &  0.013 \\
		 295.96 &  0.09 &  0.442 &  0.003 &   211 &  38 &  0.41 &  0.08 \\
		 344.30 &  0.05 &  26.59 &  0.12 &  9061 &  52 &  0.293 &  0.008 \\
		 411.14 &  0.05 &  2.238 &  0.010 &   674 &  31 &  0.259 &  0.014 \\
		 443.94 &  0.05 &  3.12 &  0.03 &   798 &  26 &  0.220 &  0.009 \\
		 778.84 &  0.06 &  12.97 &  0.06 &  1699 &  45 &  0.113 &  0.004 \\
		 867.29 &  0.08 &  4.24 &  0.02 &   477 &  34 &  0.097 &  0.007 \\
		 963.98 &  0.07 &  14.50 &  0.06 &  1686 &  36 &  0.100 &  0.003 \\
		 1085.93 &  0.08 &  10.13 &  0.06 &  1193 &  50 &  0.101 &  0.005 \\
		 1112.00 &  0.08 &  13.41 &  0.06 &  1387 &  44 &  0.089 &  0.004 \\
		 1408.14 &  0.10 &  20.85 &  0.08 &  1640 &  54 &  0.068 &  0.003 \\
	\bottomrule
	\end{tabular}
\end{table}
\FloatBarrier


\subsection{Detektoreigenschaften}
\label{sec:Detektoreigenschaften}
Um die Detektoreigenschaften genauer zu untersuchen wird das Spektrum von ${}^{137}$Cs über eine Zeit
von $t_\text{mess} = \SI{3365}{\second}$ aufgenommen. Wie in Abbildung \ref{fig:Cs_log} zu erkennen, besitzt Cäsium
ein monochromatisches Spektrum. Im Folgendem wird dieser und die beiden anderen Charakteristika, der Rückstreupeak
sowie die Comptonkante näher untersucht. Die detektierten Peaks sind in Tabelle \ref{tab:zuordnung_Cs} zugeordnet. 
Die theoretischen Emissionsenergie des Cäsium liegt bei \SI{661.657(3)}{\kilo\electronvolt} \cite{referenz1}. Mit Hilfe
ihrer, kann nach Formel \ref{eqn:Com_Kante} der theoretische Wert der Comptonkante auf $\SI{477.27}{\kilo\electronvolt}$
bestimmt werden. Des Weiteren lässt sich nach Formel \ref{eqn:rueck} auch der theoretische Wert für den Rückstreupeak
auf \SI{184.32}{\kilo\electronvolt} ermitteln. Der Wert von $m_\text{e} c^2 = \SI{510.9989461(31)}{\kilo\electronvolt}$ wird 
der Referenz \cite{codata} entnommen. 
\begin{figure}[htb]
 \centering
 \includegraphics[width=\textwidth]{build/Cs_log.pdf}
 \caption{Das aufgenommene Spektrum des ${}^{137}$Cs-Strahlers in Abhängigkeit der Energie mit logarithmierter y-Achse.}
 \label{fig:Cs_log}
\end{figure}
\begin{table}
	\centering
	\caption{Zuordnung der detektierten Peaks von Cäsium.}
	\label{tab:zuordnung_Cs}
	\begin{tabular}{
		c
		S[table-format=4.0]
		S[table-format=3.3] @{${}\pm{}$} S[table-format=1.3]
		}
	\toprule
		{} &
		{Index $i$} &
		\multicolumn{2}{c}{$E_\text{i}$\;/\;\si{\kilo\electronvolt}} \\
	\midrule
		 Rückstreupeak &  471	&	187.12 & 0.10	\\
		 Comptonkante &  1166	&	467.24 &	0.12	\\
		 Vollenergiepeak &  1649	&	661.92	& 0.13 \\
	\bottomrule
	\end{tabular}
\end{table}
Dabei wird die Energie der Comptonkante aus der Abbildung \ref{fig:Cs_log} abgelesen, ebenso wie für den 
Rückstreupeak. Für das Ablesen wir eine Genauigkeit von \num{3} Kanälen angenommen.
Für den Vollenergiepeak wir erneut wie in Abschnitt \ref{sec:Energiekalibration} ein Gauß-Fit durchgeführt.
Die Parameter ergeben sich zu
\begin{align*}
  \mu &= \SI{661.562(7)}{\kilo\electronvolt} \\
  \sigma &= \SI{1.259(10)}{\kilo\electronvolt} \\
  h &= \num{2167(15)} \\
  a &= \num{4(3)}.
\end{align*}

Eine weiter wichtige Größe des Detektors ist Halbwertsbreite $E_{\sfrac{1}{2}}$, welche ein Maß für das Auflösungsvermögen ist
und den Ge-Detektor auszeichnet. Diese, sowie die Zehntelbreite $E_{\sfrac{1}{10}}$ lassen sich aus Abbildung \ref{fig:Halb} ablesen, 
dabei ergibt sich 
\begin{gather*}
  E_{\sfrac{1}{2}} = \SI{2.20(10)}{\kilo\electronvolt} \\
  E_{\sfrac{1}{10}} = \SI{3.90(10)}{\kilo\electronvolt}.
\end{gather*}
Bei den angegebenen Fehlern handelt es sich um Ablesefehler aus der Abbildung \ref{fig:Halb}.
Der Quotient dieser beiden Größen ist ein Maß für das Energieauflösungsvermögen des Detektors. Der hier ermittelte 
Quotient der gemessenen Größen beläuft sich auf
\begin{equation*}
  \frac{E_{\sfrac{1}{10}}}{E_{\sfrac{1}{2}}} = \SI{1.77(9)}{\kilo\electronvolt}.
\end{equation*}

Die theoretischen Werte ergeben sich aus der Standardabweichung $\sigma$ des Fits. Da aber nur der Quotient interessiert, 
kürzt sich diese und es ergibt sich
\begin{equation}
  \frac{E_{\sfrac{1}{10}}}{E_{\sfrac{1}{2}}} = \frac{\sqrt{8 \cdot \log{(10)}} \cdot \sigma_\text{fit}}{\sqrt{8 \cdot \log{(2)}} \cdot \sigma_\text{fit}} = \sqrt{\frac{\log{(10)}}{\log{(2)}}} = \num{1.823}
\end{equation}
als theoretischer Wert.
\begin{figure}[htb]
 \centering
 \includegraphics[width=\textwidth]{build/vollpeak.pdf}
 \caption{Der Vollenergiepeak des Cäsium zur Bestimmung der Halbwerts-/Zehntelsbreite.}
 \label{fig:Halb}
\end{figure}

Der Vollenergiepeak kann maßgeblich durch den Photo- und Comptoneffekt entstehen. Mit Hilfe der Formel \ref{eq:Absorbtion} und 
der Extinktionskoeffizienten, welche aus Abbildung \ref{fig:Germanium} abgelesen werden, lassen sich die Absorptionswahrscheinlichkeiten
zu
\begin{align*}
  p_\text{Ph} = \SI{2.7(11)}{\percent} \\
  p_\text{Com} = \SI{75(7)}{\percent}
\end{align*}
bestimmen.
Die Extinktionskoeffizienten lassen sich auf 
\begin{gather*}
  \mu_\text{Ph} = \SI{0.007(3)}{\per\centi\meter} \\
  \mu_\text{Com} = \SI{0.37(7)}{\per\centi\meter}
\end{gather*}
bestimmen, die Länge des Detektors wurde aus Abbildung \ref{fig:Detektor} entnommen und beträgt \SI{3.9}{\centi\meter}.
Das Verhältnis der Absorptionwahrscheinlichkeiten beträgt $\num{28(12)}$.
Der Inhalt des Vollenergiepeaks lässt sich bestimmen, in dem analog zu Abschnitt \ref{sec:Vollenergienachweiseffizenz} 
vorgegangen wird. Dabei werden die auf beiden Seiten nächstliegenden \num{50} Kanäle verwendet.
Für das Comptonkontiuums wird die numerischen Integration im Bereich von
\SIrange{18.70}{467.27}{\kilo\electronvolt} verwendet. Der Inhalt beträgt für den
Vollenergiepeak \SI{3.89(7)e4}{\kilo\electronvolt} und für das Comptonkontiuum \SI{1.75(14)e5}{\kilo\electronvolt}. 
Berechnung des Verhältnis der Inhalte ergibt einen Wert von $\num{106(8)}$.
\FloatBarrier

\subsection{Aktivitätsbestimmung}
\label{sec:Aktiv}
In der Abbildung \ref{fig:mystery1} ist das Spektrum eines Strahlers dargestellt. Es soll herausgefunden werden,
ob es sich um ${}^{133}$Ba oder ${}^{125}$Sb handelt, dafür wird die Aktivität der Peaks bestimmt. Dazu werden
die Peaks dem passendem Spektrum mit ihrer Emissionswahrscheinlichkeit zugeordnet, was in Tabelle \ref{tab:zuordnung_Ba}
dargestellt ist. Der Vergleich mit der Literatur \cite{referenz1} zeigt, dass es sich um Barium handelt. 
Die Emissionswahrscheinlichkeiten werden der Referenz \cite{referenz1} entnommen um damit die Aktivität zu bestimmen. 
Des Weiteren wird der Inhalt der Peaks bestimmt und mit Hilfe der Formel 
\ref{eqn:eff} die Aktivität errechnet. Mit der Messzeit von \SI{3205}{\second} ergeben sich die Werte in Tabelle 
\ref{tab:aktivitaet_ba}. Die verwendeten Parameter des durchgeführten Gauß-Fits ergeben sich zu den in Tabelle \ref{tab:Ba} 
gegebenen Werten. Die gemittelte Aktivität ergibt sich zu
\begin{equation*}
  A = \SI{1.32(17)e3}{\becquerel}.
\end{equation*}
\begin{table}
	\centering
	\caption{Die Zuordnung zum Spektrum des ${}^{133}$Ba.}
	\label{tab:zuordnung_Ba}
	\begin{tabular}{
		S[table-format=3.4] @{${}\pm{}$} S[table-format=1.4]
		S[table-format=2.3] @{${}\pm{}$} S[table-format=1.3]
		S[table-format=3.0]
		S[table-format=3.2] @{${}\pm{}$} S[table-format=1.2]
		}
	\toprule
		\multicolumn{2}{c}{$E_\text{theo}$\;/\;\si{\kilo\electronvolt}} &
		\multicolumn{2}{c}{$W_\text{i}$\;/\;\si{\%}} &
		{$i$} &
		\multicolumn{2}{c}{$\mu_\text{fit}$\;/\;\si{\kilo\electronvolt}} \\
	\midrule
		 53.1622 &  0.0018 &  2.14 &  0.06 &  138 &  53 &  3 \\
		 79.6142 &  0.0019 &  2.63 &  0.19 &  194 &  80.93 &  0.10 \\
		 80.9979 &  0.0011 &  33.3 &  0.3 &  208 &  80.93 &  0.10 \\
		 160.6121 &  0.0016 &  0.638 &  0.006 &  405 &  155.0 &  2.4 \\
		 223.2368 &  0.0013 &  0.450 &  0.005 &  560 &  222.2 &  1.0 \\
		 276.3989 &  0.0012 &  7.13 &  0.06 &  692 &  276.41 &  0.11 \\
		 302.8508 &  0.0005 &  18.31 &  0.11 &  758 &  302.87 &  0.11 \\
		 356.0129 &  0.0007 &  62.05 &  0.19 &  890 &  356.00 &  0.11 \\
		 383.8485 &  0.0012 &  8.94 &  0.06 &  959 &  383.85 &  0.11 \\
	\bottomrule
	\end{tabular}
\end{table} 
\begin{table}
	\centering
	\caption{Berechnete Aktivität der betrachteten Emissionslinien mit dazu korrespondierenden Detektor-Effizienzen.}
	\label{tab:aktivitaet_ba}
	\begin{tabular}{
		S[table-format=2.3] @{${}\pm{}$} S[table-format=1.3]
		S[table-format=3.2] @{${}\pm{}$} S[table-format=1.2]
		S[table-format=1.4] @{${}\pm{}$} S[table-format=1.4]
		S[table-format=5.0] @{${}\pm{}$} S[table-format=2.0]
		S[table-format=4.0] @{${}\pm{}$} S[table-format=3.0]
		}
	\toprule
		\multicolumn{2}{c}{$W_\text{i}$\;/\;\si{\percent}} &
		\multicolumn{2}{c}{$E_\text{i}$\;/\;\si{\kilo\electronvolt}} &
		\multicolumn{2}{c}{$Q_\text{i}$} &
		\multicolumn{2}{c}{$Z_\text{i}$\;/\;\si{\kilo\electronvolt}} &
		\multicolumn{2}{c}{$A_\text{i}$\;/\;\si{\becquerel}} \\
	\midrule
		 0.450 &  0.005 &  223.27 &  0.20 &  1.7 &  0.7 &   109 &  48 &  1810 &  793 \\
		 18.310 &  0.110 &  302.93 &  0.05 &  0.0511 &  0.0018 &  3733 &  33 &  1708 &  41 \\
		 62.050 &  0.190 &  356.06 &  0.05 &  0.0358 &  0.0010 &  10592 &  59 &  1533 &  35 \\
		 8.940 &  0.060 &  383.91 &  0.05 &  0.072 &  0.005 &  1426 &  14 &  1482 &  37 \\
	\bottomrule
	\end{tabular}
\end{table}
\begin{table}
	\centering
	\caption{Parameter des Gauß-Fits für das gegeben Spektrum}
	\label{tab:Ba}
	\begin{tabular}{
		S[table-format=3.2] @{${}\pm{}$} S[table-format=1.2]
		S[table-format=3.2] @{${}\pm{}$} S[table-format=1.2]
		S[table-format=4.0] @{${}\pm{}$} S[table-format=2.0]
		S[table-format=1.3] @{${}\pm{}$} S[table-format=1.3]
		S[table-format=2.2] @{${}\pm{}$} S[table-format=1.2]
		}
	\toprule
		\multicolumn{2}{c}{$E_\text{i}$} &
		\multicolumn{2}{c}{$\mu_\text{i}$\;/\;\si{\kilo\electronvolt}} &
		\multicolumn{2}{c}{$h_\text{i}$} &
		\multicolumn{2}{c}{$\sigma_\text{i}$\;/\;\si{\kilo\electronvolt}} &
		\multicolumn{2}{c}{$a_\text{i}$} \\
	\midrule
		 52.99 &  0.04 &  53.16 &  0.07 &   61 &  12 &  1.1 &  0.3 &  59.1 &  1.2 \\
		 75.56 &  0.04 &  81.02 &  0.04 &  2721 &  34 &  1.55 &  0.02 &  68 &  4 \\
		 81.20 &  0.04 &  81.02 &  0.04 &  2724 &  35 &  1.55 &  0.02 &  63 &  5 \\
		 160.59 &  0.04 &  160.53 &  0.13 &   64 &  16 &  1.5 &  0.4 &  70 &  2 \\
		 223.06 &  0.04 &  223.3 &  0.2 &   21 &  6 &  2.1 &  0.7 &  27.1 &  0.9 \\
		 302.86 &  0.05 &  302.93 &  0.05 &  735 &  4 &  2.025 &  0.013 &  10.9 &  0.6 \\
		 356.06 &  0.05 &  356.06 &  0.05 &  1968 &  7 &  2.147 &  0.009 &  5.8 &  1.1 \\
		 383.86 &  0.05 &  383.91 &  0.05 &  252 &  2 &  2.257 &  0.017 &  2.5 &  0.3 \\
	\bottomrule
	\end{tabular}
\end{table}
\begin{figure}[htb]
 \centering
 \includegraphics[width=\textwidth]{build/mystery1_log.pdf}
 \caption{Das Spektrum des zu bestimmenden Strahlers(${}^{133}$Ba oder ${}^{125}$Sb).}
 \label{fig:mystery1}
\end{figure}
\FloatBarrier


\subsection{Unbekanntes Salz}
\label{sec:Salz}
Es soll nun das Spektrum eines unbekannten Strahlers untersucht werden. Hierzu wird das in Abbildung \ref{fig:Salz}
dargestellte Spektrum, welches über eine Zeit von \SI{4510}{\second} gemessen wurde näher untersucht.
Als erstes werden die auftretenden Peaks identifiziert, diese sind in Tabelle \ref{tab:Salz} dargestellt.
Mit Hilfe von \cite{referenz1} werden diese identifiziert und ihrem Emissionselement zugeordnet. Die Zuordnung ist
ebenfalls in der Tabelle \ref{tab:Salz} zu erkennen.
In Tabelle \ref{tab:aktivitaet_e} lassen sich die ermittelten Werte zur Bestimmung der Aktivität einsehen.
Die mittleren Aktivitäten der Nuklide lassen sich auf
\begin{align*}
  A_\text{Ra} & = \SI{6.3(2)e3}{\becquerel} \\
  A_\text{Pd} & = \SI{5.6(10)e3}{\becquerel} \\
  A_\text{Bi} & = \SI{3.74(17)e3}{\becquerel}
\end{align*}
bestimmen. Dabei werden nur die Werte im Bereich der \SIrange{150}{1700}{\kilo\electronvolt} verwendet, 
da nur in diesem Bereich die Effizienz nach Abschnitt \ref{sec:Vollenergienachweiseffizenz} ihre Gültigkeit besitzt.
Auf Grund der auftretenden Elemente lässt sich auf die Zerfallsreihe
von ${}^{238}$Ur schließen, was mit \cite{referenz1} bestätigt wird.
\begin{table}
	\centering
	\caption{Die ermittelten Peaks zur Nuklid Bestimmung.}
	\label{tab:Salz}
	\begin{tabular}{
		S[table-format=4.3] @{${}\pm{}$} S[table-format=1.3]
		S[table-format=2.2] @{${}\pm{}$} S[table-format=1.2]
		S[table-format=4.0]
		S[table-format=4.2] @{${}\pm{}$} S[table-format=1]
		}
	\toprule
		\multicolumn{2}{c}{$E_\text{i}$\;/\;\si{\kilo\electronvolt}} &
		\multicolumn{2}{c}{$W_\text{i}$\;/\;\si{\percent}} &
		{$i$} &
		\multicolumn{2}{c}{$E_\text{i,fit}$\;/\;\si{\kilo\electronvolt}} \\
	\midrule
		 \multicolumn{5}{c}{${}^{234}$Th} \\
		 63.300 &  0.020 &  3.75 &  0.08 &  164 &  63.47 &  0.042858 \\
		 92.380 &  0.010 &  2.18 &  0.19 &  236 &  92.48 &  0.043020 \\
		 \multicolumn{5}{c}{${}^{226}$Ra} \\
		 186.211 &  0.013 &  3.56 &  0.02 &  468 &  185.98 &  0.043925 \\
		 665.453 &  0.022 &  1.53 &  0.01 &  1657 &  665.17 &  0.056099 \\
		 1847.420 &  0.025 &  2.02 &  0.01 &  4594 &  1848.83 &  0.109521 \\
		 2118.550 &  0.030 &  1.16 &  0.01 &  5266 &  2119.65 &  0.123240 \\
		 \multicolumn{5}{c}{${}^{214}$Pb} \\
		 77.109 &  0.001 &  10.47 &  0.20 &  198 &  77.17 &  0.042927 \\
		 87.347 &  0.100 &  3.59 &  0.09 &  223 &  87.25 &  0.042986 \\
		 241.997 &  0.003 &  7.27 &  0.02 &  607 &  242.00 &  0.044737 \\
		 295.224 &  0.002 &  18.41 &  0.04 &  739 &  295.20 &  0.045684 \\
		 351.932 &  0.002 &  35.60 &  0.07 &  880 &  352.03 &  0.046873 \\
		 785.960 &  0.090 &  1.06 &  0.01 &  1956 &  785.67 &  0.060561 \\
		 1155.190 &  0.020 &  1.64 &  0.01 &  2874 &  1155.64 &  0.076187 \\
		 1238.111 &  0.012 &  5.83 &  0.01 &  3078 &  1237.85 &  0.079935 \\
		 1280.960 &  0.020 &  1.44 &  0.01 &  3184 &  1280.57 &  0.081912 \\
		 1407.980 &  0.040 &  2.39 &  0.01 &  3499 &  1407.52 &  0.087887 \\
		 1509.228 &  0.015 &  2.13 &  0.01 &  3752 &  1509.49 &  0.092780 \\
		 1661.280 &  0.060 &  1.05 &  0.01 &  4130 &  1661.83 &  0.100220 \\
		 1729.595 &  0.015 &  2.84 &  0.01 &  4300 &  1730.34 &  0.103608 \\
		 2204.210 &  0.040 &  4.91 &  0.02 &  5475 &  2203.88 &  0.127553 \\
		 \multicolumn{5}{c}{${}^{214}$Bi} \\
		 609.312 &  0.007 &  45.49 &  0.02 &  1518 &  609.15 &  0.054170 \\
		 665.453 &  0.022 &  1.53 &  0.01 &  1657 &  665.17 &  0.056099 \\
		 768.356 &  0.010 &  4.89 &  0.02 &  1912 &  767.94 &  0.059880 \\
		 934.061 &  0.012 &  3.10 &  0.01 &  2324 &  933.98 &  0.066534 \\
		 1120.287 &  0.010 &  14.91 &  0.03 &  2788 &  1120.98 &  0.074631 \\
		 1377.669 &  0.012 &  3.97 &  0.01 &  3424 &  1377.30 &  0.086451 \\
		 1764.494 &  0.014 &  15.31 &  0.05 &  4386 &  1765.00 &  0.105331 \\
	\bottomrule
	\end{tabular}
\end{table}
\begin{table}
	\centering
	\caption{Berechnete Aktivität der betrachteten Emissionslinien mit dazu korrespondierenden Detektor-Effizienzen.}
	\label{tab:aktivitaet_e}
	\begin{tabular}{
		S[table-format=4.3] @{${}\pm{}$} S[table-format=1.3]
		S[table-format=2.3] @{${}\pm{}$} S[table-format=1.3]
		S[table-format=3.0e1] @{${}\pm{}$} S[table-format=2.0e1]
		S[table-format=1.3]
		S[table-format=3.0e1] @{${}\pm{}$} S[table-format=2.0e1]
		}
	\toprule
		\multicolumn{2}{c}{$E_\text{i}$\;/\;\si{\kilo\electronvolt}} &
		\multicolumn{2}{c}{$W_\text{i}$\;/\;\si{\percent}} &
		\multicolumn{2}{c}{$Z_\text{i}$} &
		{$Q_\text{i}$} &
		\multicolumn{2}{c}{$A_\text{i}$\;/\;\si{\becquerel}} \\
	\midrule
		\multicolumn{9}{c}{\textbf{${}^{226}$Ra} }\\
		 186.211 &  0.013 &  3.555 &  0.019 &  105e1 &  20e1 &  0.213 &  20e2 &  4e2 \\
		\multicolumn{9}{c}{\textbf{${}^{214}$Pb}} \\
		 241.997 &  0.003 &  7.268 &  0.022 &  34e2 &  5e2 &  0.130 &  51e2 &  8e2 \\
		 295.224 &  0.002 &  18.414 &  0.036 &  72 &  5e2 &  0.090 &  62e2 &  4e2 \\
		 351.932 &  0.002 &  35.60 &  0.07 &  104 &  4e2 &  0.065 &  645e1 &  27e1 \\
		 785.96 &  0.09 &  1.064 &  0.013 &    32 &   3 &  0.014 &  30e2 &  3e2 \\
		 1155.190 &  0.020 &  1.635 &  0.007 &    21 &   3 &  0.007 &  27e2 &  3e2 \\
		 1238.111 &  0.012 &  5.831 &  0.014 &   91 &   6 &  0.006 & 36e2 &  2e2 \\
		 1280.960 &  0.020 &  1.435 &  0.006 &    16 &   2 &  0.006 &  28e2 &  4e2 \\
		 1407.98 &  0.04 &  2.389 &  0.008 &    24 &   3 &  0.005 &  30e2 &  3e2 \\
		 1509.228 &  0.015 &  2.128 &  0.010 &    26 &   3 &  0.004 &  41e2 &  4e2 \\
		 1661.28 &  0.06 &  1.048 &  0.009 &    11 &   1 &  0.004 &  44e2 &  5e2 \\
		\multicolumn{7}{c}{\textbf{${}^{214}$Bi}} \\
		 609.312 &  0.007 &  45.49 &  0.19 &  417e1 &  9e1 &  0.023 &  563e1 &  18e1 \\
		 665.453 &  0.022 &  1.530 &  0.007 &    47 &   6 &  0.020 &  22e2 &  3e2 \\
		 768.356 &  0.010 &  4.892 &  0.016 &   210 &  19 &  0.015 &  41e2 &  4e2 \\
		 934.061 &  0.012 &  3.100 &  0.010 &    73 &   6 &  0.010 &  322e1 &  29e1 \\
		 1120.287 &  0.010 &  14.91 &  0.03 &   344 &  16 &  0.007 &  443e1 &  22e1 \\
		 1377.669 &  0.012 &  3.968 &  0.011 &   53 &   3 &  0.005 & 376e1 &  24e1 \\
	\bottomrule
	\end{tabular}
\end{table}
\begin{figure}[htb]
 \centering
 \includegraphics[width=\textwidth]{build/Uran.pdf}
 \caption{Das Spektrum eines unbekannten Nuklids.}
 \label{fig:Salz}
\end{figure}

% \subsection{Unterkapiel}
% \label{sec:Unterkapitel}

% \begin{figure}
%   \centering
%   \includegraphics[width=\textwidth]{Plot.pdf}
%   \caption{Bildunterschrift}
%   \label{fig:Plot1}
% \end{figure}
