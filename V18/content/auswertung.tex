\newpage
\section{Auswertung}
\label{sec:Auswertung}

\subsection{Energiekalibration}
\label{sec:Energiekalibration}
Zu Beginn des Versuchs wurde zunächst die Probe ${}^{152}$ Eu untersucht. Die Messwerte
sind in Abbildung \ref{fig:Eu_log_Kanal} logaritmisch dargestellt.
Die für die Kalibrierung verwendeten Peaks sind mit roten Kreuzen makiert. Die Messwerte
wurden mit den theoretischen Werten verglichen, sodass mit Hilfe einer linearen Ausgleichsrechnung
eine Zuordnung von Kanal und Energie möglich ist. Die dafür verwendeten Daten sind
gemeinsam mit den Emissionswahrscheinlichkeit in Tabelle \ref{tab:kalibration}
gelistet, dabei wurden Energien mit einer Emissionswahrscheinlichkeit $W$
kleiner als \SI{2}{\percent} vernachlässigt. Für die Ausgleichsrechnung wurde
eine Gerade der Form
\begin{equation}
  E = m \cdot i + b
\end{equation}
verwendet, wobei die Steigung $m$ zu \SI{0.40303(7)}{\kilo\electronvolt} und der
y-Achsenabschnitt $b$ zu \SI{-2.7(1)}{\kilo\electronvolt} bestimmt wurde.

In der Abbildung \ref{fig:kalibration} ist das Ergebnis der Ausgleichsrechnung graphisch
dargestellt, dabei befindet sich auf der x-Achse die Kanalnummer und auf der
y-Achse die errechnete Energie.

\begin{figure}
 \centering
 \includegraphics[width=\textwidth]{Eu_log_Kanal.pdf}
 \caption{Die Zählrate $N$ in Abhängigkeit der Kanalnummer $i$ für ${}^{152}$ Eu mit logaritmirter y-Skala.}
 \label{fig:Eu_log_Kanal}
\end{figure}

\begin{figure}
 \centering
 \includegraphics[width=\textwidth]{build/kalibration.pdf}
 \caption{Die ermittelte Energie anhand des Fits in Abhängigkeit der Kanalnummer.}
 \label{fig:kalibration}
\end{figure}

\begin{table}
	\centering
	\caption{Gegebene Werte zur Kalibrierung des Germanium-Detektors \cite{referenz1}.}
	\label{tab:zuordnung_eu}
	\begin{tabular}{
		S[table-format=4.4] @{${}\pm{}$} S[table-format=1.4]
		S[table-format=2.3] @{${}\pm{}$} S[table-format=1.3]
		S[table-format=4.0]
		}
	\toprule
		\multicolumn{2}{c}{ $E_\text{i}$\;/\;\si{\kilo\electronvolt}} &
		\multicolumn{2}{c}{ $W_\text{i}$\;/\;\si{\percent}} &
		{$i$} \\
	\midrule
		 121.7817 &  0.0003 &  28.41 &  0.13 &  309 \\
		 244.6974 &  0.0008 &  7.55 &  0.04 &  614 \\
		 295.9387 &  0.0017 &  0.442 &  0.003 &  740 \\
		 344.2785 &  0.0012 &  26.59 &  0.12 &  861 \\
		 411.1165 &  0.0012 &  2.238 &  0.010 &  1027 \\
		 443.965 &  0.003 &  3.120 &  0.028 &  1108 \\
		 778.9045 &  0.0024 &  12.97 &  0.06 &  1939 \\
		 867.380 &  0.003 &  4.243 &  0.023 &  2159 \\
		 964.079 &  0.018 &  14.50 &  0.06 &  2399 \\
		 1085.837 &  0.010 &  10.13 &  0.06 &  2702 \\
		 1112.076 &  0.003 &  13.41 &  0.06 &  2765 \\
		 1408.013 &  0.003 &  20.85 &  0.08 &  3500 \\
	\bottomrule
	\end{tabular}
\end{table}
\begin{table}
	\centering
	\caption{Parameter des durchgeführten Gauss-Fits pro Kanal.}
	\label{tab:gauss_parameter}
	\begin{tabular}{
		S[table-format=4.0]
		S[table-format=4.3] @{${}\pm{}$} S[table-format=1.3]
		S[table-format=2.1] @{${}\pm{}$} S[table-format=1.1]
		S[table-format=4.0] @{${}\pm{}$} S[table-format=2.0]
		S[table-format=1.3] @{${}\pm{}$} S[table-format=1.3]
		}
	\toprule
		{$i$} &
		\multicolumn{2}{c}{$\mu_\text{i}$} &
		\multicolumn{2}{c}{$a_\text{i}$} &
		\multicolumn{2}{c}{$h_\text{i}$} &
		\multicolumn{2}{c}{$\sigma_\text{i}$} \\
	\midrule
		 309 &  308.773 &  0.005 &  98.8 &  2.6 &  5344 &  20 &  1.644 &  0.007 \\
		 614 &  613.758 &  0.015 &  44.6 &  1.1 &  757 &  7 &  1.941 &  0.022 \\
		 740 &  740.872 &  0.187 &  31.4 &  0.7 &   43 &  5 &  1.971 &  0.269 \\
		 861 &  860.839 &  0.007 &  24.8 &  1.0 &  1668 &  6 &  2.167 &  0.009 \\
		 1027 &  1026.680 &  0.054 &  20.1 &  0.6 &  118 &  3 &  2.283 &  0.079 \\
		 1108 &  1108.071 &  0.042 &  18.9 &  0.5 &  134 &  3 &  2.376 &  0.060 \\
		 1939 &  1939.059 &  0.046 &  15.2 &  0.7 &  202 &  3 &  3.351 &  0.067 \\
		 2159 &  2158.525 &  0.137 &  14.4 &  0.5 &   51 &  2 &  3.709 &  0.201 \\
		 2399 &  2398.434 &  0.046 &  8.1 &  0.5 &  160 &  2 &  4.206 &  0.068 \\
		 2702 &  2701.024 &  0.102 &  7.1 &  0.7 &  101 &  3 &  4.707 &  0.150 \\
		 2765 &  2765.719 &  0.080 &  6.2 &  0.6 &  115 &  2 &  4.825 &  0.118 \\
		 3500 &  3500.536 &  0.093 &  1.1 &  0.7 &  119 &  3 &  5.498 &  0.139 \\
	\bottomrule
	\end{tabular}
\end{table}

Die in Tabelle \ref{tab:kalibration} gelisteten Peaks wurden mit Hilfe der
Gauss-Funktion der Form
\begin{equation}
  g(i) = h \cdot \exp{\left(\frac{\left(i - \mu\right)^2}{2 \cdot \sigma^2}\right)} + a
\end{equation}
gefittet. Die Ergebnisse dieses Fits sind in Tabelle \ref{tab:gauss_parameter}
dargestellt, dabei bezeichnet $\mu$ den Mittelwert, $h$ die Höhe des Peaks, $\sigma$
die Standardabweichung und $a$ ein Parameter, welcher den Untergrund darstellt.
\begin{figure}
 \centering
 \includegraphics[width=\textwidth]{build/efficiency.pdf}
 \caption{Die Vollenergienachweiseffizenz des Detektors gegen die Energie aufgetragen.}
 \label{fig:effizenz}
\end{figure}


\subsection{Vollenergienachweiseffizenz}
\label{sec:Vollenergienachweiseffizenz}
Die Messzeit betrug \SI{ 4676}{\second}. Der Abstand Probe zum Detektor betrug
\SI{9.8(1)}{\centi\meter}, dabei wurde die Probe \SI{7.3}{\centi\meter} oberhalb
des Aluminumgehäuses befestigt. Von der Aluminiumhaube zur Probe sind es dann noch
einmal \SI{1.5}{\centi\meter}.

\subsection{Detektoreigenschaften}
\label{sec:Detektoreigenschaften}

\begin{table}
	\centering
	\caption{Peakinhalt, Energie und Detektoreffizenz als Ergebnis des Gaußfits.}
	\label{tab:det_eff}
	\begin{tabular}{
		S[table-format=4.2] @{${}\pm{}$} S[table-format=1.2]
		S[table-format=2.3] @{${}\pm{}$} S[table-format=1.3]
		S[table-format=5.0] @{${}\pm{}$} S[table-format=3.0]
		S[table-format=1.3] @{${}\pm{}$} S[table-format=1.3]
		}
	\toprule
		\multicolumn{2}{c}{$E_\text{i}$\;/\;\si{\kilo\electronvolt}} &
		\multicolumn{2}{c}{$W_\text{i}$\;/\;\si{\percent}} &
		\multicolumn{2}{c}{$Z_\text{i}$\;/\;\si{\kilo\electronvolt}} &
		\multicolumn{2}{c}{$Q_\text{i}$\;/\;\si{\becquerel }} \\
	\midrule
		 121.81 &  0.04 &  28.41 &  0.13 &  22027 &  124 &  0.666 &  0.018 \\
		 244.73 &  0.05 &  7.55 &  0.04 &  3685 &  55 &  0.419 &  0.013 \\
		 295.96 &  0.09 &  0.442 &  0.003 &   211 &  38 &  0.41 &  0.08 \\
		 344.30 &  0.05 &  26.59 &  0.12 &  9061 &  52 &  0.293 &  0.008 \\
		 411.14 &  0.05 &  2.238 &  0.010 &   674 &  31 &  0.259 &  0.014 \\
		 443.94 &  0.05 &  3.12 &  0.03 &   798 &  26 &  0.220 &  0.009 \\
		 778.84 &  0.06 &  12.97 &  0.06 &  1699 &  45 &  0.113 &  0.004 \\
		 867.29 &  0.08 &  4.24 &  0.02 &   477 &  34 &  0.097 &  0.007 \\
		 963.98 &  0.07 &  14.50 &  0.06 &  1686 &  36 &  0.100 &  0.003 \\
		 1085.93 &  0.08 &  10.13 &  0.06 &  1193 &  50 &  0.101 &  0.005 \\
		 1112.00 &  0.08 &  13.41 &  0.06 &  1387 &  44 &  0.089 &  0.004 \\
		 1408.14 &  0.10 &  20.85 &  0.08 &  1640 &  54 &  0.068 &  0.003 \\
	\bottomrule
	\end{tabular}
\end{table}

% \subsection{Unterkapiel}
% \label{sec:Unterkapitel}

% \begin{figure}
%   \centering
%   \includegraphics[width=\textwidth]{Plot.pdf}
%   \caption{Bildunterschrift}
%   \label{fig:Plot1}
% \end{figure}
