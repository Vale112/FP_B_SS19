\newpage
\section{Auswertung}
\label{sec:Auswertung}

\subsection{Energiekalibration}
\label{sec:Energiekalibration}
Zu Beginn des Versuchs wurde zunächst die Probe ${}^{152}$ Eu untersucht. Die Messwerte
sind in Abbildung \ref{fig:Eu_log_Kanal} logaritmisch dargestellt.
Die für die Kalibrierung verwendeten Peaks sind mit roten Kreuzen makiert. Die Messwerte
wurden mit den theoretischen Werten verglichen, sodass mit Hilfe einer linearen Ausgleichsrechnung
eine Zuordnung von Kanal und Energie möglich ist. Die dafür verwendeten Daten sind
gemeinsam mit den Emissionswahrscheinlichkeit in Tabelle \ref{tab:zuordnung_eu}
gelistet, dabei wurden Energien mit einer Emissionswahrscheinlichkeit $W$
kleiner als \SI{2}{\percent} vernachlässigt. Für die Ausgleichsrechnung wurde
eine Gerade der Form
\begin{equation}
  E = m \cdot i + b
\end{equation}
verwendet, wobei $m$ die Steigung und $b$ der y-Achsenabschnitt ist. Dabei liesen sich die Parameter zu
\begin{gather}
  m = \SI{0.40303(7)}{\kilo\electronvolt\per\kanal} \\
  b = \SI{-2.7(1)}{\kilo\electronvolt}
\end{gather}
bestimmen.
In der Abbildung \ref{fig:kalibration} ist das Ergebnis der Ausgleichsrechnung graphisch
dargestellt, dabei befindet sich auf der x-Achse die Kanalnummer und auf der
y-Achse die errechnete Energie.

\begin{figure}[htb]
 \centering
 \includegraphics[width=\textwidth]{Eu_log_Kanal.pdf}
 \caption{Die Zählrate $N$ in Abhängigkeit der Kanalnummer $i$ für ${}^{152}$Eu mit logaritmirter y-Skala.}
 \label{fig:Eu_log_Kanal}
\end{figure}

\begin{figure}[htb]
 \centering
 \includegraphics[width=\textwidth]{build/kalibration.pdf}
 \caption{Die ermittelte Energie anhand der Daten in Abhängigkeit der Kanalnummer.}
 \label{fig:kalibration}
\end{figure}

\begin{table}
	\centering
	\caption{Gegebene Werte zur Kalibrierung des Germanium-Detektors \cite{referenz1}.}
	\label{tab:zuordnung_eu}
	\begin{tabular}{
		S[table-format=4.4] @{${}\pm{}$} S[table-format=1.4]
		S[table-format=2.3] @{${}\pm{}$} S[table-format=1.3]
		S[table-format=4.0]
		}
	\toprule
		\multicolumn{2}{c}{ $E_\text{i}$\;/\;\si{\kilo\electronvolt}} &
		\multicolumn{2}{c}{ $W_\text{i}$\;/\;\si{\percent}} &
		{$i$} \\
	\midrule
		 121.7817 &  0.0003 &  28.41 &  0.13 &  309 \\
		 244.6974 &  0.0008 &  7.55 &  0.04 &  614 \\
		 295.9387 &  0.0017 &  0.442 &  0.003 &  740 \\
		 344.2785 &  0.0012 &  26.59 &  0.12 &  861 \\
		 411.1165 &  0.0012 &  2.238 &  0.010 &  1027 \\
		 443.965 &  0.003 &  3.120 &  0.028 &  1108 \\
		 778.9045 &  0.0024 &  12.97 &  0.06 &  1939 \\
		 867.380 &  0.003 &  4.243 &  0.023 &  2159 \\
		 964.079 &  0.018 &  14.50 &  0.06 &  2399 \\
		 1085.837 &  0.010 &  10.13 &  0.06 &  2702 \\
		 1112.076 &  0.003 &  13.41 &  0.06 &  2765 \\
		 1408.013 &  0.003 &  20.85 &  0.08 &  3500 \\
	\bottomrule
	\end{tabular}
\end{table}


\subsection{Vollenergienachweiseffizenz}
\label{sec:Vollenergienachweiseffizenz}
Die Aktivität der Probe am $15.04.2019$, dem Versuchstag lässt sich mit
\begin{equation}
  A = A_0 \cdot \exp{\left(-\lambda \cdot tv\right)}
\end{equation}
berechnen, hierbei ist $A_0$ die Aktivität am Produktionstag, $\lambda$ die Zerfallskonstante und $t$ die Zeit.
Es ist aus \cite{V18} bekannt, dass die Probe am Produktionstag, dem
$01.10.2000$ eine Aktivität von $\SI{4130(60)}{\becquerel}$ besaß.
Die Aktivität am Versuchstag beläuft sich damit auf
\begin{equation}
  \SI{1599(23)}{\becquerel}.
\end{equation}
Des Weiteren lässt sich mit Hilfe der Formel \ref{eqn:raum} der Raumwinkel errechnen.
Dabei beträgt $a~=~\SI{8.8(1)}{\centi\meter}$, was sich aus dem Abstand der Probe zum
Aluminiumgehäuse (\SI{7.3(1)}{\centi\meter}) und dem Abstand des Gehäuses zum Detektor 
(\SI{1.5}{\centi\meter}) ergibt. Der Radius $r$ des Detektors betrug \SI{2.25}{\centi\meter}, was sich
aus der Abbildung \ref{fig:Detektor} ablesen lässt. Damit lässt sich der Raumwinkel zu
\begin{equation}
  \frac{\Omega}{4\pi} = \num{0.0156(3)}
\end{equation}
bestimmen.

Die in Tabelle \ref{tab:zuordnung_eu} gelisteten Peaks wurden mit Hilfe der
Gauss-Funktion der Form
\begin{equation}
  g(i) = h \cdot \exp{\left(\frac{\left(i - \mu\right)^2}{2 \cdot \sigma^2}\right)} + a
\end{equation}
gefittet. Die Ergebnisse dieses Fits sind in Tabelle \ref{tab:gauss_parameter}
dargestellt, dabei bezeichnet $\mu$ den Mittelwert, $h$ die Höhe des Peaks, $\sigma$
die Standardabweichung und $a$ ein Parameter, welcher den Untergrund darstellt.
Unter zu Hilfe nahme von
\begin{equation}
  Z_\text{i} = \sqrt{2\pi} h_\text{i} \sigma_{i}
\end{equation}
lässt sich der Peakinhalt $Z_\text{i}$ bestimmen. Nun lässt sich mit Formel \ref{eqn:eff}
die Nachweiswahrscheinlichkeit des Detektors errechnen. Dabei ist die Messzeit von $t_\text{mess} = \SI{4676}{\second}$
zu berücksichtigen. Die sich ergeben Werte sind in Tabelle \ref{tab:det_eff} gelistet.

Anhand der Werte aus Tabelle \ref{tab:det_eff} lässt sich ein Fit der Form
\begin{equation}
  Q(E) = a \cdot \left(E-b \right)^c + d
\end{equation}
durchführen. Die Parametre ergeben sich dabei zu
\begin{gather}
  a = \num{0.01(5)} \\
  b = \num{121.8(6)} \\
  c = \num{0.5(6)} \\
  d = \num{0.43(31)}.
\end{gather}
Eine graphische Darstellung dieser Ausgleichsrechnung ist in Abbildung \ref{fig:effizenz} zu finden.

\begin{figure}[htb]
 \centering
 \includegraphics[width=\textwidth]{build/efficiency.pdf}
 \caption{Die Vollenergienachweiseffizenz des Detektors gegen die Energie aufgetragen.}
 \label{fig:effizenz}
\end{figure}

\begin{table}
	\centering
	\caption{Parameter des durchgeführten Gauss-Fits pro Kanal.}
	\label{tab:gauss_parameter}
	\begin{tabular}{
		S[table-format=4.0]
		S[table-format=4.3] @{${}\pm{}$} S[table-format=1.3]
		S[table-format=2.1] @{${}\pm{}$} S[table-format=1.1]
		S[table-format=4.0] @{${}\pm{}$} S[table-format=2.0]
		S[table-format=1.3] @{${}\pm{}$} S[table-format=1.3]
		}
	\toprule
		{$i$} &
		\multicolumn{2}{c}{$\mu_\text{i}$} &
		\multicolumn{2}{c}{$a_\text{i}$} &
		\multicolumn{2}{c}{$h_\text{i}$} &
		\multicolumn{2}{c}{$\sigma_\text{i}$} \\
	\midrule
		 309 &  308.773 &  0.005 &  98.8 &  2.6 &  5344 &  20 &  1.644 &  0.007 \\
		 614 &  613.758 &  0.015 &  44.6 &  1.1 &  757 &  7 &  1.941 &  0.022 \\
		 740 &  740.872 &  0.187 &  31.4 &  0.7 &   43 &  5 &  1.971 &  0.269 \\
		 861 &  860.839 &  0.007 &  24.8 &  1.0 &  1668 &  6 &  2.167 &  0.009 \\
		 1027 &  1026.680 &  0.054 &  20.1 &  0.6 &  118 &  3 &  2.283 &  0.079 \\
		 1108 &  1108.071 &  0.042 &  18.9 &  0.5 &  134 &  3 &  2.376 &  0.060 \\
		 1939 &  1939.059 &  0.046 &  15.2 &  0.7 &  202 &  3 &  3.351 &  0.067 \\
		 2159 &  2158.525 &  0.137 &  14.4 &  0.5 &   51 &  2 &  3.709 &  0.201 \\
		 2399 &  2398.434 &  0.046 &  8.1 &  0.5 &  160 &  2 &  4.206 &  0.068 \\
		 2702 &  2701.024 &  0.102 &  7.1 &  0.7 &  101 &  3 &  4.707 &  0.150 \\
		 2765 &  2765.719 &  0.080 &  6.2 &  0.6 &  115 &  2 &  4.825 &  0.118 \\
		 3500 &  3500.536 &  0.093 &  1.1 &  0.7 &  119 &  3 &  5.498 &  0.139 \\
	\bottomrule
	\end{tabular}
\end{table}
\begin{table}
	\centering
	\caption{Peakinhalt, Energie und Detektoreffizenz als Ergebnis des Gaußfits.}
	\label{tab:det_eff}
	\begin{tabular}{
		S[table-format=4.2] @{${}\pm{}$} S[table-format=1.2]
		S[table-format=2.3] @{${}\pm{}$} S[table-format=1.3]
		S[table-format=5.0] @{${}\pm{}$} S[table-format=3.0]
		S[table-format=1.3] @{${}\pm{}$} S[table-format=1.3]
		}
	\toprule
		\multicolumn{2}{c}{$E_\text{i}$\;/\;\si{\kilo\electronvolt}} &
		\multicolumn{2}{c}{$W_\text{i}$\;/\;\si{\percent}} &
		\multicolumn{2}{c}{$Z_\text{i}$\;/\;\si{\kilo\electronvolt}} &
		\multicolumn{2}{c}{$Q_\text{i}$\;/\;\si{\becquerel }} \\
	\midrule
		 121.81 &  0.04 &  28.41 &  0.13 &  22027 &  124 &  0.666 &  0.018 \\
		 244.73 &  0.05 &  7.55 &  0.04 &  3685 &  55 &  0.419 &  0.013 \\
		 295.96 &  0.09 &  0.442 &  0.003 &   211 &  38 &  0.41 &  0.08 \\
		 344.30 &  0.05 &  26.59 &  0.12 &  9061 &  52 &  0.293 &  0.008 \\
		 411.14 &  0.05 &  2.238 &  0.010 &   674 &  31 &  0.259 &  0.014 \\
		 443.94 &  0.05 &  3.12 &  0.03 &   798 &  26 &  0.220 &  0.009 \\
		 778.84 &  0.06 &  12.97 &  0.06 &  1699 &  45 &  0.113 &  0.004 \\
		 867.29 &  0.08 &  4.24 &  0.02 &   477 &  34 &  0.097 &  0.007 \\
		 963.98 &  0.07 &  14.50 &  0.06 &  1686 &  36 &  0.100 &  0.003 \\
		 1085.93 &  0.08 &  10.13 &  0.06 &  1193 &  50 &  0.101 &  0.005 \\
		 1112.00 &  0.08 &  13.41 &  0.06 &  1387 &  44 &  0.089 &  0.004 \\
		 1408.14 &  0.10 &  20.85 &  0.08 &  1640 &  54 &  0.068 &  0.003 \\
	\bottomrule
	\end{tabular}
\end{table}
\FloatBarrier


\subsection{Detektoreigenschaften}
\label{sec:Detektoreigenschaften}
Um die Detektoreigenschaften genauer zu untersuchen wurde das Spektrum von ${}^{137}$Cs über eine Zeit
von $t_\text{mess} = \SI{3365}{\second}$ aufgenommen. Wie in Abbildung \ref{fig:Cs_log} zu erkennen, besitzt Cäsium
ein monochromatisches Spektrum. Im Folgendem wird dieser und die beiden anderen charakteristischen Peaks, der Rückstreupeak
sowie die Comptonkante näher untersucht. Die detektierten Peaks sind in Tabelle \ref{tab:zuordnung_Cs} zugeordnet worden. 
Die theoretischen Emissionsenergie des Cäsium liegt bei \SI{661.657(3)}{\kilo\electronvolt} \cite{referenz1}. Mit Hilfe
ihrer, kann nach Formel \ref{eqn:Com_Kante} der theoretische Wert der Comptonkante auf $\SI{477.27}{\kilo\electronvolt}$
bestimmt werden. Des Weiteren lässt sich nach Formel \ref{eqn:rueck} auch der theoretische Wert für den Rückstreupeak
auf \SI{184.32} ermitteln.
\begin{figure}[htb]
 \centering
 \includegraphics[width=\textwidth]{build/Cs_log.pdf}
 \caption{Das augenommen Spektrum des ${}^{137}$Cs-Stahlers in Abhängigkeit der Energie.}
 \label{fig:Cs_log}
\end{figure}
\begin{table}
	\centering
	\caption{Zuordnung der detektierten Peaks von Cäsium.}
	\label{tab:zuordnung_Cs}
	\begin{tabular}{
		c
		S[table-format=4.0]
		S[table-format=3.2] @{${}\pm{}$} S[table-format=1.3]
		}
	\toprule
		{} &
		{Index $i$} &
		\multicolumn{2}{c}{$E_\text{i, ist}$\;/\;\si{\kilo\electronvolt}} \\
	\midrule
		 Rückstreupeak &  471	&	187.20 & 1.4	\\
		 Comptonkante &  1166	&	467.29 &	1.4	\\
		 Vollenergiepeak &  1649	&	661.562	& 0.007\\
	\bottomrule
	\end{tabular}
\end{table}
Eine weiter wichtige Größe des Detektors ist Halbwertsbreite $E_{\sfrac{1}{2}}$, welche ein Maß für das Auflösungsverhalten ist
und den Ge-Detektor auszeichnet. Diese, sowie die Zehntelbreite $E_{\sfrac{1}{10}}$ lassen sich aus Abbildung \ref{fig:Halb} ablesen, 
dabei ergibt sich 
\begin{gather}
  E_{\sfrac{1}{2}} = \SI{2.2(1)}{\kilo\electronvolt} \\
  E_{\sfrac{1}{10}} = \SI{3.9(1)}{\kilo\electronvolt}.
\end{gather}
Der Quotient dieser beiden Größen ist ein Maß für das Energieauflösungsvermögen des Detektors. Der hier ermittelte 
Quotient
\begin{equation}
  \frac{E_{\sfrac{1}{10}}}{E_{\sfrac{1}{2}}} = \num{1.77(9)}
\end{equation}
der gemessenen Größen stehen im Vergleich zum theoretischen Wert \cite{V18} von
\begin{equation}
  \frac{E_{\sfrac{1}{10}}}{E_{\sfrac{1}{2}}} = \num{1.82}.
\end{equation}
\begin{figure}[htb]
 \centering
 \includegraphics[width=\textwidth]{build/vollpeak.pdf}
 \caption{Der Vollenergiepeak des Cäsium zur Bestimmung der Halbwerts-/Zehntelsbreite.}
 \label{fig:Halb}
\end{figure}
Der Vollenergiepeak kann maßgeblich durch den Photo- und Comptoneffekt entstehen. Mit Hilfe der Formel \ref{eq:Absorbtion} und 
der Extinktionskoeffizienten, welche aus Abbildung \ref{fig:Germanium} abgelesen werden, lassen sich die Absorbtionwahrscheinlichkeiten
zu
\begin{align}
  p_\text{Ph} = \SI{3(1)}{\percent} \\
  p_\text{Com} = \SI{75(7)}{\percent}.
\end{align}
Die Extinktionskoeffizienten liesen sich zuvor auf 
\begin{gather}
  \mu_\text{Ph} = \SI{0.007(3)}{\per\centi\meter} \\
  \mu_\text{Com} = \SI{0.037(7)}{\per\centi\meter}
\end{gather}
bestimmen, die Länge des detektors wurde aus Abbildung \ref{fig:Detektor} entnommen und betrug \SI{3.9}{\centi\meter}.

Der Inhalt des Vollenergiepeaks, sowie des Comptonkontiuums lässt sich analog zu Abschnitt \ref{sec:Vollenergienachweiseffizenz}
bestimmen. Dabei beträgt dieser für den Vollenergiepeak \SI{3.89(7)e4}{\kilo\electronvolt} und für das Comptonkontiuum 
\SI{1.75(14)e5}{\kilo\electronvolt}.


\subsection{mystery1}
\label{sec:}
Peak bei ca. 275 keV (Kanal 692) kp woher
Messzeit \SI{3205}{\second}

\subsection{Unbekanntes Salz}
\label{sec:Salz}
Messzeit \SI{4510}{\second}

% \subsection{Unterkapiel}
% \label{sec:Unterkapitel}

% \begin{figure}
%   \centering
%   \includegraphics[width=\textwidth]{Plot.pdf}
%   \caption{Bildunterschrift}
%   \label{fig:Plot1}
% \end{figure}
