\newpage
\section{Auswertung}
\label{sec:Auswertung}

Zu Beginn des Versuchs wurde zunächst die Probe ${}^{152}$ Eu untersucht. Die Messwerte
sind in Abbildung \ref{fig:Eu_log_Kanal} logaritmisch dargestellt. Die Messwerte
wurden mit den theoretischen Werten verglichen, sodass mit Hilfe einer linearen Ausgleichsrechnung
eine Zuordnung von Kanal und Energie möglich ist. Die dafür verwendeten Daten sind
in Tabelle \ref{tab:kalibration} gelistet und in der Abbildung \ref{fig:kalibration}
gemeinsam mit der Ausgleichsgeraden eingetragen.

\begin{figure}
 \centering
 \includegraphics[width=\textwidth]{Eu_log_Kanal.pdf}
 \caption{Die Zählrate $N$ in Abhängigkeit der Kanalnummer $i$ für ${}^{152}$ Eu mit logaritmirter y-Skala.}
 \label{fig:Eu_log_Kanal}
\end{figure}

\begin{figure}
 \centering
 \includegraphics[width=\textwidth]{build/kalibration.pdf}
 \caption{Die ermittelte Energie anhand des Fits in Abhängigkeit der Kanalnummer.}
 \label{fig:kalibration}
\end{figure}

\begin{figure}
 \centering
 \includegraphics[width=\textwidth]{build/efficiency.pdf}
 \caption{Die Vollenergienachweiseffizenz des Detektors gegen die Energie aufgetragen.}
 \label{fig:effizenz}
\end{figure}

\begin{table}
	\centering
	\caption{Gegebene Werte zur Kalibrierung des Germanium-Detektors \cite{referenz1}.}
	\label{tab:zuordnung_eu}
	\begin{tabular}{
		S[table-format=4.4] @{${}\pm{}$} S[table-format=1.4]
		S[table-format=2.3] @{${}\pm{}$} S[table-format=1.3]
		S[table-format=4.0]
		}
	\toprule
		\multicolumn{2}{c}{ $E_\text{i}$\;/\;\si{\kilo\electronvolt}} &
		\multicolumn{2}{c}{ $W_\text{i}$\;/\;\si{\percent}} &
		{$i$} \\
	\midrule
		 121.7817 &  0.0003 &  28.41 &  0.13 &  309 \\
		 244.6974 &  0.0008 &  7.55 &  0.04 &  614 \\
		 295.9387 &  0.0017 &  0.442 &  0.003 &  740 \\
		 344.2785 &  0.0012 &  26.59 &  0.12 &  861 \\
		 411.1165 &  0.0012 &  2.238 &  0.010 &  1027 \\
		 443.965 &  0.003 &  3.120 &  0.028 &  1108 \\
		 778.9045 &  0.0024 &  12.97 &  0.06 &  1939 \\
		 867.380 &  0.003 &  4.243 &  0.023 &  2159 \\
		 964.079 &  0.018 &  14.50 &  0.06 &  2399 \\
		 1085.837 &  0.010 &  10.13 &  0.06 &  2702 \\
		 1112.076 &  0.003 &  13.41 &  0.06 &  2765 \\
		 1408.013 &  0.003 &  20.85 &  0.08 &  3500 \\
	\bottomrule
	\end{tabular}
\end{table}
\begin{table}
	\centering
	\caption{Parameter des durchgeführten Gauss-Fits pro Kanal für das Spektrum von ${}^{152}$Eu.}
	\label{tab:gauss_parameter}
	\begin{tabular}{
		S[table-format=4.0]
		S[table-format=4.3] @{${}\pm{}$} S[table-format=1.3]
		S[table-format=2.1] @{${}\pm{}$} S[table-format=1.1]
		S[table-format=4.0] @{${}\pm{}$} S[table-format=2.0]
		S[table-format=1.3] @{${}\pm{}$} S[table-format=1.3]
		}
	\toprule
		{$i$} &
		\multicolumn{2}{c}{$\mu_\text{i}$} &
		\multicolumn{2}{c}{$a_\text{i}$} &
		\multicolumn{2}{c}{$h_\text{i}$} &
		\multicolumn{2}{c}{$\sigma_\text{i}$} \\
	\midrule
		 309 &  308.773 &  0.005 &  99 &  3 &  5344 &  20 &  1.644 &  0.007 \\
		 614 &  613.758 &  0.015 &  44.6 &  1.1 &  757 &  7 &  1.94 &  0.02 \\
		 740 &  740.872 &  0.187 &  31.4 &  0.7 &   43 &  5 &  1.97 &  0.27 \\
		 861 &  860.839 &  0.007 &  24.8 &  1.0 &  1668 &  6 &  2.167 &  0.009 \\
		 1027 &  1026.680 &  0.054 &  20.1 &  0.6 &  118 &  3 &  2.28 &  0.08 \\
		 1108 &  1108.071 &  0.042 &  18.9 &  0.5 &  134 &  3 &  2.38 &  0.06 \\
		 1939 &  1939.059 &  0.046 &  15.2 &  0.7 &  202 &  3 &  3.35 &  0.07 \\
		 2159 &  2158.525 &  0.137 &  14.4 &  0.5 &   51 &  2 &  3.8 &  0.2 \\
		 2399 &  2398.434 &  0.046 &  8.1 &  0.5 &  160 &  2 &  4.21 &  0.07 \\
		 2702 &  2701.024 &  0.102 &  7.1 &  0.7 &  101 &  3 &  4.71 &  0.15 \\
		 2765 &  2765.719 &  0.080 &  6.2 &  0.6 &  115 &  2 &  4.83 &  0.12 \\
		 3500 &  3500.536 &  0.093 &  1.1 &  0.7 &  119 &  3 &  5.50 &  0.14 \\
	\bottomrule
	\end{tabular}
\end{table}
\begin{table}
	\centering
	\caption{Die Werte zur Bestimmung der Detektoreffizenz als Ergebnis des Gaußfits.}
	\label{tab:det_eff}
	\begin{tabular}{
		S[table-format=4.2] @{${}\pm{}$} S[table-format=1.2]
		S[table-format=2.3] @{${}\pm{}$} S[table-format=1.3]
		S[table-format=5.0] @{${}\pm{}$} S[table-format=3.0]
		S[table-format=1.3] @{${}\pm{}$} S[table-format=1.3]
		}
	\toprule
		\multicolumn{2}{c}{$E_\text{i}$\;/\;\si{\kilo\electronvolt}} &
		\multicolumn{2}{c}{$W_\text{i}$\;/\;\si{\percent}} &
		\multicolumn{2}{c}{$Z_\text{i}$\;/\;\si{\kilo\electronvolt}} &
		\multicolumn{2}{c}{$Q_\text{i}$\;/\;\si{\becquerel }} \\
	\midrule
		 121.81 &  0.04 &  28.41 &  0.13 &  22027 &  124 &  0.666 &  0.018 \\
		 244.73 &  0.05 &  7.55 &  0.04 &  3685 &  55 &  0.419 &  0.013 \\
		 295.96 &  0.09 &  0.442 &  0.003 &   211 &  38 &  0.41 &  0.08 \\
		 344.30 &  0.05 &  26.59 &  0.12 &  9061 &  52 &  0.293 &  0.008 \\
		 411.14 &  0.05 &  2.238 &  0.010 &   674 &  31 &  0.259 &  0.014 \\
		 443.94 &  0.05 &  3.12 &  0.03 &   798 &  26 &  0.220 &  0.009 \\
		 778.84 &  0.06 &  12.97 &  0.06 &  1699 &  45 &  0.113 &  0.004 \\
		 867.29 &  0.08 &  4.24 &  0.02 &   477 &  34 &  0.097 &  0.007 \\
		 963.98 &  0.07 &  14.50 &  0.06 &  1686 &  36 &  0.100 &  0.003 \\
		 1085.93 &  0.08 &  10.13 &  0.06 &  1193 &  50 &  0.101 &  0.005 \\
		 1112.00 &  0.08 &  13.41 &  0.06 &  1387 &  44 &  0.089 &  0.004 \\
		 1408.14 &  0.10 &  20.85 &  0.08 &  1640 &  54 &  0.068 &  0.003 \\
	\bottomrule
	\end{tabular}
\end{table}


% \subsection{Unterkapiel}
% \label{sec:Unterkapitel}

% \begin{figure}
%   \centering
%   \includegraphics[width=\textwidth]{Plot.pdf}
%   \caption{Bildunterschrift}
%   \label{fig:Plot1}
% \end{figure}
