\newpage
\section{Auswertung}

Zu Beginn des Versuchs wurde zunächst die Probe {}^152 Eu untersucht. Die Messwerte
sind in Abbildung \ref{fig:Eu_log} logaritmisch dargestellt. Die Messwerte
wurden mit den theoretischen Werten verglichen, sodass mit Hilfe einer linearen Ausgleichsrechnung
eine Zuordnung von Kanal und Energie möglich ist. Die dafür verwendeten Daten sind
in Tabelle \ref{tab:Kalibration} gelistet und in der Abbildung \ref{fig:Kalibration}
gemeinsam mit der Ausgleichsgeraden eingetragen.

\begin{figure}
 \centering
 \includegraphics[width=\textwidth]{Eu_log.pdf}
 \caption{Die Zählrate $N$ in Abhängigkeit der Kanalnummer für {}^152 Eu mit logaritmirter y-Skala.}
 \label{fig:Eu_log}
\end{figure}


\label{sec:Auswertung}
% \subsection{Unterkapiel}
% \label{sec:Unterkapitel}

% \begin{figure}
%   \centering
%   \includegraphics[width=\textwidth]{Plot.pdf}
%   \caption{Bildunterschrift}
%   \label{fig:Plot1}
% \end{figure}
