\section{Diskussion}
\label{sec:Diskussion}
Zunächst ist anzumerken, dass in allen Abbildungen nur ein Ausschnitt
der Messdaten dargestellt ist. Der Ausschnitt wurde so gewählt, dass alle
relevanten Informationen enthalten sind.

Im Abschnitt \ref{sec:Energiekalibration} lässt sich in Abbildung \ref{fig:kalibration}
erkennen, dass die Ausgleichsrechnung für die Zuordnung von Kanal zu Energie
sehr gut passt. Alle Punkte liegen auf der Geraden, die Fehlerbalken der
verwendeten Daten sind nicht direkt zu erkennen, da die Unsicherheiten im Vergleich zu
den Werten deutlich kleiner sind. Dies ist auch in den folgenden Abbildungen
der Fall. Ebenso lässt sich dies auch an den kleinen Unsicherheiten der Parameter erkennen.
Auch bei den Parametern der Gauß-Fits für die Peaks, welche in Tabelle
\ref{tab:gauss_parameter} dargestellt sind, sind die Unsicherheiten sehr klein.

Die Abweichung von \SI{1.51(6)}{\percent} für die
Position des Rückstreupeaks des ${}^{137}$Cs-Spektrums vom
berechneten Wert \SI{184.32}{\kilo\electronvolt} ist nur sehr gering. Noch
geringer ist die Abweichung für die Bestimmung der Energie des Vollenergiepeaks
von nur \SI{1.95(21)e-4}{\percent}. Es ergibt sich allerdings eine
Abweichung von etwa \SI{-2.115(24)}{\percent} für die Position
der Compton-Kante.
Das Verhältnis von der Halbwertsbreite zur Zehntelsbreite weicht auch 
um \SI{-3(4)}{\percent} vom erwarteten Wert \num{1.823} ab. Damit ist
also die Annahme von Gauß-Kurven-förmigen Peaks gerechtfertigt.
Aus dem Verhältnis der Absorptionswahrscheinlichkeiten lässt sich darauf schließen,
dass das $\gamma$-Quant zunächst durch den Compton-Effekt seine Energie reduziert,
um anschließend vollständig mit Hilfe des Photoeffekts seine Energie abzugeben.
Dies ist möglich, da im linken Bereich der Abbildung \ref{fig:Germanium}
die Wahrscheinlichkeit für den Photoeffekt die des Comptoneffekts
übertrifft.

Die Aktivität besitzt einen realistischen Wert, da diese in der Größenordnung der Aktivität von
${}^{152}$Eu liegt. Lässt man den ersten Wert aus der Mittlung raus, da dieser deutlich
kleiner als die anderen ist, ergibt sich eine Aktivität von \SI{228(8)e1}{\becquerel}.

Bei der Untersuchung des unbekannten Strahlers in Abschnitt \ref{sec:Salz}
lässt sich auf Grund der auftretenden Emissionslinien auf die Zerfallsreihe
von ${}^{238}$U schließen. Das Spektrum enthält einige gut erkennbare
Peaks, welche alle den angegebenen Strahlern zugeordnet werden können.
In Tabelle \ref{tab:aktivitaet_e} wird keine Unsicherheit für $Q_\text{i}$
angegeben, da diese sich im Bereich \num{e-12} und kleiner befinden.
Ebenso werden zu zwei Energie werten keine Unsicherheiten angegeben, da
hier in der Literatur keine Angaben zu gemacht wurden.