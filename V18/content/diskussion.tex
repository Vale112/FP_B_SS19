\section{Diskussion}
\label{sec:Diskussion}
Zunächst ist anzumerken, dass in allen Abbildungen nur ein Ausschnitt
der Messdaten dargestellt ist. Der Ausschnitt wurde so gewählt, dass alle
relevanten Informationen enthalten sind.

Im Abschnitt \ref{sec:Energiekalibration} lässt sich in Abbildung \ref{fig:kalibration}
erkennen, dass die Ausgleichsrechnung für die Zuordnung von Kanal zu Energie
sehr gut passt. Alle Punkte liegen auf der Geraden, die Fehlerbalken der
verwendeten Daten sind nicht direkt zu erkennen, da die Fehler im Vergleich zu
den Werten deutlich kleiner sind. Dies ist auch in den folgenden Abbildungen
der Fall. Ebenso lässt sich dies auch an den kleinen Fehlern der Parameter erkennen.
Auch bei den Parametern der Gauß-Fits für die Peaks, welche in Tabelle
\ref{tab:gauss_parameter} dargestellt sind, sind die Fehler sehr klein.

In Abschnitt \ref{sec:Vollenergienachweiseffizenz} lässt sich für die Ausgleichsrechnung nach
der Potenzfunktion \ref{eqn:Potenz} nur eine Konvergenz erzwingen, wenn der erste Wert
berücksichtigt wird, obwohl er unter der Grenze von \SI{150}{\kilo\electronvolt}
liegt, ab der die Quanten die Hülle und äußere Schicht des Detektors nahezu ungehindert 
durchdringen. Wie in Abbildung \ref{fig:effizenz} und an den Fehlern der 
Parametern, welche meist größer als der Wert sind, zu erkennen ist, handelt es sich
eher um eine Näherung des Verhaltens der Messdaten.

Die Abweichung des ermittelten Wertes \SI{187.2(14)}{\kilo\electronvolt} der
Position des Rückstreupeaks des ${}^{137}$Cs-Spektrums vom
berechneten Wert \SI{184.32}{\kilo\electronvolt} ist nur sehr gering. Noch
geringer ist die Abweichung für die Bestimmung der Energie des Vollenergiepeaks
von nur \SI{0.95}{\kilo\electronvolt}. Es ergibt sich allerdings eine
größere Abweichung von etwa \SI{10}{\kilo\electronvolt} für die Position
der Comptonkante.
Das Verhältnis von der Halbwertsbreite zur Zehntelsbreite weicht auch nur 
um \SI{-2.9}{\percent} vom erwarteten Wert \num{1.823} ab. Damit ist
also die Annahme von Gauß-Kurven-förmigen Peaks gerechtfertigt.
Aus dem Verhältnis der Absorptionswahrscheinlichkeiten lässt sich darauf schließen,
dass das $\gamma$-Quant zunächst durch den Compton-Effekt seine Energie reduziert
um anschließend vollständig mit Hilfe des Photoeffekts seine Energie abgibt.
Dies ist möglich, da im linken Bereich der Abbildung \ref{fig:Germanium}
die Wahrscheinlichkeit für den Photoeffekt, die des Comptoneffekts
übertrifft.

In Abbildung \ref{fig:mystery1} lässt sich das Spektrum von 
${}^{133}$Ba gut erkennen, allerdings lässt sich ein makanter 
Peak bei etwa \SI{275}{\kilo\electronvolt} keinem Nuklid 
mit Hilfe von \cite{referenz1} zuordnen. Die Aktivität besitzt auch
einen realistischen Wert.

Bei der Untersuchung des unbekannten Strahlers in Abschnitt \ref{sec:Salz}
lässt sich auf Grund der auftretenden Emissionslinien auf die Zerfallsreihe
von ${}^{138}$Ur schließen. Das Spektrum enthält einige gut erkennbare
Peaks, welche alle den angegbenen Strahlern zugeordnet werden können.
In Tabelle \ref{tab:aktivitaet_e} wird kein Fehler für $Q_\text{i}$
angegeben, da diese sich im Bereich \num{e-17} und kleiner befinden.